\documentclass[a4paper,12pt]{article}
\usepackage{amsfonts}
\usepackage{tcolorbox}

\usepackage{mathtools}
\DeclarePairedDelimiter{\ceil}{\lceil}{\rceil}

\usepackage{hyperref}
\hypersetup{
	colorlinks=true,
	linkcolor=blue
}

\begin{document}

\title{Awkward Number Series}
\author{Will Dengler}
\maketitle

\section{Notation and Assumed Knowledge}
\label{section:notation_and_assumed_knowledge}


\label{notation}
\hypertarget{notation}{}
\begin{tcolorbox}
\textbf{Notation}

\begin{itemize}

\item $\mathbb{Z}$ is defined to be the set of integers.

\item $\mathbb{N} \subset \mathbb{Z}$ is defined to be the set of natural numbers, including $0$.

\item $\mathbb{N}^+ \subset \mathbb{N}$ is defined to be the set of positive integers.

\item For any $x \in \mathbb{N}^+$, $[x] = \{$ $j \in \mathbb{N}$ $|$ $j < x$ $\}$.

\item $\mathbb{Q}$ is defined to be the set of rational numbers.
\end{itemize}
\end{tcolorbox}
\noindent \\









\label{assumed_knowledge:divisibility_uniqueness}
\hypertarget{assumed_knowledge:divisibility_uniqueness}{}
\begin{tcolorbox}
\textbf{Assumed Knowledge}

\noindent For any natural number $x$, for any positive integer $y$, there exists a unique integers $z \in \mathbb{N}$ and $r \in [y]$ such that $x = zy + r$. We call $r$ the \textit{remainder} of $x$ when divided by $z$.
\end{tcolorbox}
\noindent \\






\label{definition:remainder_function}
\hypertarget{definition:remainder_function}{}
\begin{tcolorbox}
\textbf{Definition}

\noindent For any natural number $x$, for any positive integer $y$,  the remainder function $\rho : (\mathbb{N} \times \mathbb{N}^+) \rightarrow \mathbb{N}$, $\rho(x, y)$ is defined to be the remainder of $x$ when divided by $y$.
\end{tcolorbox}
\noindent \\






\label{definition:divisibility}
\hypertarget{definition:divisibility}{}
\begin{tcolorbox}
\textbf{Definition}

\noindent For any natural number $x$, for any positive integer $y$, if $\rho(x,y) = 0$, then we say that $x$ is \textit{divisible} by $y$.
\end{tcolorbox}
\noindent \\






\label{assumed_knowledge:remainder_function}
\hypertarget{definition:remainder_function}{}
\begin{tcolorbox}
\textbf{Remainder Function Properties}

\noindent The remainder function has the following properties:

\begin{itemize}

\item For any $i \in \mathbb{N}$, for any $j \in [i]$, $\rho(j, i) = j$.

\item For any $i, j, k \in \mathbb{N}$, $\rho(kj, ki) = k \rho(j, i)$.

\item For any $i, j, k \in \mathbb{N}$, $\rho(j + k, i) = \rho(\rho(j, i) + \rho(k, i), i)$. 

\end{itemize}
\end{tcolorbox}
\noindent \\






\label{definition:ceiling_function}
\hypertarget{definition:ceiling_function}{}
\begin{tcolorbox}
\textbf{Definition}

\noindent For any $q \in \mathbb{Q}$, the \textit{ceiling function} $\ceil{q} = z$, where $z$ is the integer such that $z - 1 < q \leq z$.
\end{tcolorbox}
\noindent \\




\label{lemma:ceiling_function}
\hypertarget{lemma:ceiling_function}{}
\begin{tcolorbox}
\textbf{Lemma}

\noindent For any $q = \frac{a}{b} \in \mathbb{Q}$:

\begin{itemize}
\item $\ceil{q} = q$ whenever $\rho(a, b) = 0$.

\item  $\ceil{q} = \frac{c}{b}$, where $c = a + b - \rho(a, b)$ whenever $\rho(a, b) > 0$.
\end{itemize}


\end{tcolorbox}
\noindent \\






\label{definition:gcd}
\hypertarget{definition:gcd}{}
\begin{tcolorbox}
\textbf{Definition}

\noindent For any $x, y \in \mathbb{N}$, the function $gcd(x, y)$ is defined to be the greatest common divisor of $x$ and $y$.
\end{tcolorbox}
\noindent \\







\label{assumed_knowledge:recursive_prime_definition}
\hypertarget{assumed_knowledge:recursive_prime_definition}{}
\begin{tcolorbox}
\textbf{Assumed Knowledge}

\noindent The prime numbers can be recursively defined as the series:

\begin{itemize}

\item $p_0 = 2$ is the first element in the series.

\item For all $k \in \mathbb{N}^+$, $p_k$ is the least greatest integer such that $p_k > p_{k-1}$, and for all $j < k$, $p_k$ is not divisible by $p_j$.

\end{itemize}
\end{tcolorbox}
\noindent \\







\section{Awkward Number Series}
\label{section:awkward_number_series}


\label{definition:awkward_number_series}
\hypertarget{definition:awkward_number_series}{}
\begin{tcolorbox}
\textbf{Definition}

\noindent For any positive integers $a, n$, the \textit{awkward number series}, $S_{a, n}$ is defined as:

\begin{itemize}
\item An initial element $s_0 = a + n$
\item For any $i > 0$, $s_i$ is defined to be the least greatest integer such that $s_i > s_{i - 1}$ and $\rho(s_i, s_k) \geq a$ so all $k < i$.
\end{itemize}


\noindent We say that the awkward number series $S_{a,n}$ has $a$ \textit{activators}, and $n$ \textit{initial non-activators}.

\end{tcolorbox}
\noindent \\





\label{lemma:prime_asn}
\hypertarget{lemma:prime_asn}{}
\begin{tcolorbox}
\textbf{Lemma}

The awkward number series $S_{1, 1}$ is equal to the set of prime numbers.
\end{tcolorbox}

\noindent \\
\textit{Proof}

\noindent TODO

\begin{center}
\noindent\rule{8cm}{0.4pt}
\end{center}







\label{theorem:infinite_asn}
\hypertarget{theorem:infinite_asn}{}
\begin{tcolorbox}
\textbf{Awkward Infinity Theorem}

Every awkward number series contains an infinite number of elements.
\end{tcolorbox}

\noindent \\
\textit{Proof}

\noindent Let $S_{a,n}$ be any awkward number series.\\

\noindent Assume that $S_{a,n}$ contains a finite number of elements.\\

\noindent Let $s_i$ be the greatest element within $S_{a, n}$.\\

\noindent Let $m$ be any positive common multiple of the elements of $S_{a, n}$.\\

\noindent Notice that $m > s_i$ since $m$ is a multiple of $s_i$, but $s_i$ is not a multiple of any $s_j$ such that $j < i$.\\

\noindent Consider the value $m + a$.\\

\noindent Since $S_{a, n}$ is finite, there must exist some element, $s_j \in S_{a,n}$ such that $\rho(m + a, s_j) < a$. Otherwise, there is some element smaller than $m + a$ that has not been accounted for, or $m + a$ would be an element of $S_{a, n}$ that has not been accounted for.\\

\noindent Let $\rho(m + a, s_j) = b$.\\

\noindent There exists some integer $x$ such that $m + a = xs_j + b$.\\

\noindent Since $m$ is a common multiple of all the elements of $S_{a,n}$, then $\frac{m}{s_j} \in \mathbb{N}$.\\

\noindent Let $y = \frac{m}{s_j}$. Then $m = ys_j$.\\

\noindent Consider the equation $a = (m + a) - m$.\\

\noindent Substituting $xs_j + b$ for $m + a$ yields $a = xs_j + b - m$.\\

\noindent Substituting $ys_j$ for $m$ yields $a = xs_j + b - ys_j$.\\

\noindent Applying the distributive property yields $a = (x - y)s_j + b$.\\

\noindent If $x < y$, then $(x - y)s_j \leq -s_j$.\\

\noindent Since $0 \leq b < a < s_j$, then $(x - y)s_j + b < 0$ if $x < y$.\\

\noindent However, $a > 0$, as such, $x < y$ cannot be the case.\\

\noindent If $x > y$, then $(x - y)s_j \geq s_j$.\\

\noindent Since $0 \leq b$ and $a < s_j$, then $(x - y)s_j + b \geq s_j$ if $x > y$.\\

\noindent However, $a < s_j$, as such, $x > y$ cannot be the case.\\

\noindent As such, $x = y$ must be the case.\\

\noindent Substituting $x$ for $y$ yields $a = (x - x)s_j + b = b$.\\

\noindent By assumption, $b < a$, as such we have reached a contradiction.\\

\noindent Therefore, it must be the case that either $m + a$ is an element of $S_{a,n}$, or there exists some other element in $S_{a,n}$ less than $m + a$ that was not accounted for. In either case, $S_{a,n}$ cannot be finite.

\begin{center}
\noindent\rule{8cm}{0.4pt}
\end{center}







\label{corollary:infinite_primes}
\hypertarget{corollary:infinite_primes}{}
\begin{tcolorbox}
\textbf{Corollary}

There are an infinite number of prime numbers.
\end{tcolorbox}

\noindent \\
\textit{Proof}

\noindent The prime numbers are an awkward number series and every awkward number series contains an infinite number of elements.

\begin{center}
\noindent\rule{8cm}{0.4pt}
\end{center}







\label{lemma:initial_staples}
\hypertarget{lemma:initial_staples}{}
\begin{tcolorbox}
\textbf{Lemma}

For any awkward number series $S_{a,n}$, the first $\ceil{\frac{n}{a}} + 1$ elements are given by $s_i = a(i + 1) + n$.
\end{tcolorbox}

\noindent \\
\textit{Proof}

\noindent Let $S_{a,n}$ be any awkward number series. We shall complete this proof via induction on the index of the first $\ceil{\frac{n}{a}} + 1$ elements.\\


\noindent
\textit{Base Case}

\noindent By definition, the initial element is $s_0 = a + n = a(0 + 1) + n$.\\


\noindent
\textit{Inductive Hypothesis}

\noindent Assume for the first $0 \leq j < \ceil{\frac{n}{a}}$, that $s_j = a(j + 1) + n$.\\


\noindent
\textit{Inductive Step}

\noindent For all $x \in [a]$, $\rho(s_j + x, s_j) = x < a$. As such, $s_{j+1} \geq s_j + a$.\\

\noindent If we can show that $\rho(s_j + a, s_k) \geq a$ for all $k < j$, then $s_{j+1} = s_j + a$.\\

\noindent Furthermore, $s_j + a = a(j + 1) + n + a = a(j + 2) + n$, thus we will completed our proof.\\

\noindent Let $0 \leq k < j$. Then $s_j + a = s_k + a(j - k + 1)$ according to the inductive hypothesis.\\

\noindent As such, $\rho(s_j + a, s_k) = a(j - k + 1)$ as long as $a(j - k + 1) < s_k$.\\

\noindent Since $j < \ceil{\frac{n}{a}}$ and $j \in \mathbb{N}$, then $j \leq \ceil{\frac{n}{a}} - 1$.\\

\noindent As such, $a(j - k + 1) \leq a(\ceil{\frac{n}{a}} - 1 - k + 1) = a(\ceil{\frac{n}{a}} - k) = a\ceil{\frac{n}{a}} - ak$.\\

\noindent First, let us consider the case where $a$ $|$ $n$.\\

\noindent We will then have $a \ceil{\frac{n}{a}} = n$.\\

\noindent As such, $a(j - k + 1) \leq n - ak < s_k$.\\

\noindent Now let us consider the case where $\rho(n, a) \geq 1$.\\\

\noindent Then $a \ceil{\frac{n}{a}} = a \frac{n + a - \rho(n, a)}{a} = n + a - \rho(n, a) = s_0 - \rho(n, a) < s_0 \leq s_k$.\\

\noindent As such, $a(j - k + 1) < s_k - ak \leq s_k$.\\

\noindent Therefore, $\rho(s_j + a, s_k) = a(j - k + 1)$ does in fact hold.\\

\noindent As such, we now need to show that $a(j - k + 1) \geq a$.\\

\noindent We chose $k < j$, as such, $a(j - k + 1) \geq a(j - j + 1) = a$.\\

\noindent We have shown that $s_j + a = a(j + 2) + n$ is the least greatest integer greater than $s_j$ such that $\rho(s_j + a, s_k) \geq a$ for all $k \leq j$. Therefore, $s_{j+1} = a(j + 2) + n$.

\begin{center}
\noindent\rule{8cm}{0.4pt}
\end{center}








\label{definition:awkward_number_series}
\hypertarget{definition:awkward_number_series}{}
\begin{tcolorbox}
\textbf{Definition}

For any awkward number series $S_{a,n}$, the value $\ceil{\frac{n}{a}} + 1$ is called the \textit{dimension} of the series.

\end{tcolorbox}
\noindent \\








\label{lemma:relation_to_other}
\hypertarget{lemma:relation_to_other}{}
\begin{tcolorbox}
\textbf{Lemma}

For any awkward number series $S_{a,n}$, for any $i > 0$, there exists $s_j < s_i$ such that $\rho(s_i, s_j) = a$.

\end{tcolorbox}

\noindent \\
\textit{Proof}


\begin{center}
\noindent\rule{8cm}{0.4pt}
\end{center}









\label{theorem:awkward_linearity}
\hypertarget{theorem:awkward_linearity}{}
\begin{tcolorbox}
\textbf{Awkward Linearity Theorem}

For any awkward number series $S_{a,n}$, for any $s_i \in S_{a,n}$, there exists integers $x, y \in \mathbb{N}^+$ such that $s_i = xa + yn$.

\end{tcolorbox}

\noindent \\
\textit{Proof}

\noindent This shall be a proof by induction. Let $S_{a,n}$ be any awkward number series.


\noindent \\
\textit{Base Case}

\noindent By definition, $s_0 = a + n = 1a + 1n$.


\noindent \\
\textit{Inductive Hypothesis}

\noindent Assume for some $0 \leq k$, that $s_i = xa + yn$ for some $x, y \in \mathbb{N}^+$ whenever $i \leq k$.


\noindent \\
\textit{Inductive Step}

\noindent By previous lemma, there exists some $s_i < s_{k+1}$ and some $t \in \mathbb{N}^+$ such that $s_{k+1} = ts_i + a$.\\

\noindent By the inductive hypothesis, $s_i = xa + yn$ for some integers $x, y \in \mathbb{N}^+$.\\

\noindent Substituting for $s_i$ yields, $s_{k+1} = t(xa + yn) + a = txa + a + yn = (tx + 1)a + yn$.\\



\begin{center}
\noindent\rule{8cm}{0.4pt}
\end{center}
\noindent \\






\label{corollary:relation_to_initial}
\hypertarget{corollary:relation_to_initial}{}
\begin{tcolorbox}
\textbf{Corollary}

For any awkward number series $S_{a,n}$, for any $s_0 < s_i \in S_{a,n}$, there exists integers $t, r \in \mathbb{N}^+$ such that $s_i = ts_0 + ra$.

\end{tcolorbox}


\noindent \\
\textit{Proof}

\noindent Let $S_{a, n}$ be any awkward number series. We shall complete this proof by induction.


\noindent \\
\textit{Base Case}

\noindent By previous lemma $s_1 = 2a + n = (a + n) + a = s_0 + a$.


\noindent \\
\textit{Inductive Hypothesis}

\noindent Assume for some $1 \leq k$, that $s_i = ts_0 + ra$ for some integers $t, r \in \mathbb{N}^+$ whenever $i \leq k$.


\noindent  \\
\textit{Inductive Step}

\noindent By previous lemma, there exists some $s_i < s_{k+1}$ and some $t \in \mathbb{N}^+$ such that $s_{k+1} = ts_i + a$.\\


\noindent If $s_i = s_0$, then we would have $s_{k+1} = ts_0 + a$. As such, we would have nothing left to show.\\

\noindent Let us assume $s_i > s_0$.\\

\noindent By inductive hypothesis, $s_i = us_0 + va$ for some integers $u, v \in \mathbb{N}^+$.\\

\noindent Substituting for $s_i$ yields, $s_{k + 1} = t(us_0 + va) + a = tus_0 + a(tv + 1)$


\begin{center}
\noindent\rule{8cm}{0.4pt}
\end{center}
\noindent \\







\label{corollary:relation_to_initial_p2}
\hypertarget{corollary:relation_to_initial_p2}{}
\begin{tcolorbox}
\textbf{Corollary}

For any awkward number series $S_{a,n}$, for any $s_0 < s_i \in S_{a,n}$, there exists integers $t, r \in \mathbb{N}^+$ such that $s_i = (t + r)a + tn$.

\end{tcolorbox}


\noindent \\
\textit{Proof}

\noindent Let $S_{a,n}$ be any awkward number series. Let $s_0 < s_i \in S_{a, n}$.\\

\noindent By previous corollary, $s_i = ts_0 + ra$ for some integers $t, r \in \mathbb{N}^+$.\\

\noindent Substituting for $a + n$ for $s_0$ yields, $s_i = t(a + n) + ra$.\\

\noindent Distributing $t$ over $a + n$ yields, $s_i = ta + ra + tn = (t + r)a + tn$.

\begin{center}
\noindent\rule{8cm}{0.4pt}
\end{center}
\noindent \\







\label{theorem:awkward_uniqueness}
\hypertarget{theorem:awkward_uniqueness}{}
\begin{tcolorbox}
\textbf{Awkward Uniqueness Theorem}

\noindent For any two awkward number series $S_{a,b}$ and $S_{c,d}$, $S_{a,b} = S_{c,d}$ if and only if $a = c$ and $b = d$.\\

\noindent In other words, no two awkward series contain the same set of elements.
\end{tcolorbox}

\noindent \\
\textit{Proof}

\noindent Let $S_{a,n}$ be any awkward number series. Assume $S_{c, d} = S_{a, n}$ for some awkward number series $S_{c,d}$.\\


\noindent Let us refer to the elements of $S_{a,n}$ as $s_0, s_1, ...$, and the elements of $S_{c,d}$ by $s^*_0, s^*_1, ...$.\\

\noindent By definition, $s_0 = a + n$, and $s^*_0 = c + d$.\\

\noindent By assumption, $s_0 = s^*_0$. As such, $a + n = c + d$.\\

\noindent Solving for $c$ yields, $c = a + n - d$.\\

\noindent By previous lemma, $s_1 = 2a + n$, and $s^*_1 = 2c + d$.\\

\noindent By assumption, $s_1 = s^*_1$. As such, $2a + n = 2c + d$.\\

\noindent Substituting $c = a + n - d$ yields, $2a + n = 2(a + n - d) + d$.\\

\noindent Distributing the $2$ yields, $2a + n = 2a + 2n - 2d + d = 2a + 2n - d$.\\

\noindent Subtracting the $d$ from both sides yields, $2a + n + d = 2a + 2n$.\\

\noindent Subtracting the $2a$ from both sides yields $n + d = 2n$.\\

\noindent Subtracting $n$ from both sides yields $d = n$.\\

\noindent Substituting $n$ for $d$ into $a + n = c + d$ yields $a + n = c + n$.\\

\noindent Subtracting $n$ from both sides yields $a = c$.

\begin{center}
\noindent\rule{8cm}{0.4pt}
\end{center}










\label{definition:simple_and_redundant_series}
\hypertarget{definition:simple_and_redundant_series}{}
\begin{tcolorbox}
\textbf{Definition}

An awkward number series, $S_{a,n}$ is called \textit{simple} if the $gcd(a,n) = 1$. Otherwise the awkward number series is said to be \textit{redundant}.

\end{tcolorbox}
\noindent \\









\label{definition:similar_series}
\hypertarget{definition:similar_series}{}
\begin{tcolorbox}
\textbf{Definition}

Any two awkward number series $S_{a,b}$ and $S_{c,d}$ are called \textit{similar} whenever $\frac{a}{gcd(a,b)} = \frac{c}{gcd(c,d)}$ and $\frac{b}{gcd(a,b)} = \frac{d}{gcd(c,d)}$. Otherwise the series are said to be \textit{dissimilar}.

\end{tcolorbox}
\noindent \\







\label{theorem:similar_theorem}
\hypertarget{theorem:similar_theorem}{}
\begin{tcolorbox}
\textbf{Awkward Similarity Theorem}

\noindent For any simple awkward number series $S_{a, n}$, for any positive integer $x$, the elements of the awkward number series $S_{xa, xn} = \{$ $xs_i$ $|$ $s_i \in S_{a,n}$ $\}$.
\end{tcolorbox}

\noindent \\
\textit{Outline}

\noindent This will be a proof by induction on the index of the elements.

\noindent \\
\textit{Proof}

\noindent Let $S_{a,n}$ be any simple awkward number series. Let $j$ be any positive integer.\\

\noindent We shall denote the elements of $S_{a,n}$ as $s_0, s_1, ...$. We will denote the elements of $S_{ja, jn}$ as $s^*_0, s^*_1, ...$.

\noindent \\
\textit{Base Case}

\noindent By definition, the first element of $S_{ja, jn}$ is $s^*_0 = ja + jn = j(a + n)$.\\

\noindent By definition, the first element of $S_{a,n}$ is $s_0 = a + n$.\\

\noindent As such, $s^*_0 = j(a + n) = js_0$.



\noindent \\
\textit{Inductive Hypothesis}

\noindent Assume for all $0 \leq i$ that $s^*_i = js_i$.


\noindent \\
\textit{Inductive Step}

\noindent We shall start by showing that $\rho(js_{i+1}, s^*_k) \geq ja$ for all $k \leq i$. Afterwards, we will then show that $js_{i+1}$ is the least greatest integer that is both greater than $s^*_i$ with this property. As such, $s^*{i+1} = js^{i+1}$ by definition.

\noindent By the inductive hypothesis, $s^*_k = js_k$ for all $k \leq i$.\\

\noindent As such, $\rho(js_{i+1}, s^*_k) = \rho(js_{i+1}, js_k)$.\\

\noindent By previous lemma (TODO), $\rho(js_{i+1}, js_k) = j \rho(s_{i+1}, s_k)$.\\

\noindent By definition, $\rho(s_{i+1}, s_k) \geq a$ for all $k \leq i$.\\

\noindent As such, $\rho(js_{i+1}, s^*_k) = j \rho(s_{i+1}, s_k) \geq ja$.\\

\noindent Thus, we have shown that $js^{i+1}$ is a viable element of $S_{ja, jn}$. We now must show that that $js^{i+1}$ is the least greatest integer greater than $s^*_i$ with the divisibility property.\\

\noindent Assume there exists some integer $s^*_i < x < js^{i+1}$ such that $\rho(x, s^*_k) \geq ja$ for all $k \leq i$.\\

\noindent We know that $x = tj + \rho(x, j)$, for some $x \in \mathbb{N}$.

\noindent Let $r = \rho(x, j)$. Then $x = tj + r$.\\

\noindent Let $k \in [i + 1]$. Then $\rho(x, s^*_k) = \rho(tj + r, s^*_k)$.\\

\noindent By the inductive hypothesis, $s^*_k = js_k$.\\

\noindent As such, $\rho(tj + r, s^*_k) = \rho(tj + r, js_k)$\\

\noindent By remainder property (TODO), $\rho(tj + r, js_k) = \rho(\rho(tj, js_k) + \rho(r, js_k), js_k)$.\\

\noindent By remainder property (TODO), $\rho(tj, js_k) = j \rho(t, s_k)$.\\

\noindent Since $r < j < js_k$, then $\rho(r, js_k) = r$.\\

\noindent As such, $\rho(\rho(tj, js_k) + \rho(r, js_k), js_k) = \rho(j \rho(t, s_k) + r, js_k)$.\\

\noindent By definition, $0 \leq \rho(t, s_k) < s_k$. Furthermore, $\rho(t, s_k) \in \mathbb{N}$. As such, $\rho(t, s_k) \leq s_k - 1$.\\

\noindent As such, $j \rho(t, s_k) \leq j(s_k - 1)$.\\

\noindent Thus, $j \rho(t, s_k) + r \leq j(s_k - 1) + r$.\\

\noindent We also know that $r < j$.\\

\noindent As such, $j(s_k - 1) + r < j(s_k - 1) + j = j(s_k - 1 + 1) = js_k$.\\

\noindent As such, $j \rho(t, s_k) + r < js_k$, thus $\rho(x, s^*_k) = \rho(j \rho(t, s_k) + r, js_k) = j \rho(t, s_k) + r$.\\

\noindent By assumption, $\rho(x, s^*_k) \geq ja$.\\

\noindent As such, $j \rho(t, s_k) + j) > j \rho(t, s_k) + r \geq ja$.\\

\noindent As such, $j(\rho(t, s_k) + 1) > ja$.\\

\noindent As such, $\rho(t, s_k) + 1 > a$.\\

\noindent Thus, $\rho(t, s_k) \geq a$.\\

\noindent Now if we can show that $s_i < t < s_{i+1}$, then $t$ would have to be element $s_{i + 1} \in S_{a, n}$ which would be a contradiction.\\

\noindent By assumption, $s^*_i < x = jt + r$.\\

\noindent $s^*_i = js_i$ by the inductive hypothesis.\\

\noindent As such, $js_i < jt + r < jt + j = j(t + 1)$\\

\noindent Thus $s_i < t + 1$. Since $s_i \in \mathbb{N}$, then $s_i \leq t$.\\

\noindent However, we've shown that $\rho(t, s_i) \geq a > 0$. As such, $t \neq s_i$. Thus, $s_i < t$ must be the case.\\

\noindent Now we just need to show that $t < js_{i + 1}$.\\

\noindent We know that $x = tj + r < js_{i+1}$.\\

\noindent As such, $tj \leq tj + r < js_{i+1}$. Thus, $t < s_{i + 1}$.\\

\noindent But this would mean that $t$ must be the $(i + 2)^{th}$ element of $S_{a, n}$, which is a contradiction.


\begin{center}
\noindent\rule{8cm}{0.4pt}
\end{center}


\end{document}






