\documentclass[a4paper,12pt]{article}
\usepackage{amsfonts}
\usepackage{tcolorbox}

\usepackage{mathtools}
\DeclarePairedDelimiter{\ceil}{\lceil}{\rceil}

\usepackage{hyperref}
\hypersetup{
	colorlinks=true,
	linkcolor=blue
}

\begin{document}

\title{Awkward Number Series}
\author{Will Dengler}
\maketitle

\section{Notation and Assumed Knowledge}
\label{section:notation_and_assumed_knowledge}


\subsection{Notation}
\label{subsection:notation}

\label{notation}
\hypertarget{notation}{}
\begin{tcolorbox}
\textbf{Notation}

\begin{itemize}

\item $\mathbb{Z}$ is defined to be the set of integers.

\item $\mathbb{N} \subset \mathbb{Z}$ is defined to be the set of natural numbers, including $0$.

\item $\mathbb{N}^+ \subset \mathbb{N}$ is defined to be the set of positive integers.

\item For any $x \in \mathbb{N}^+$, $[x] = \{$ $j \in \mathbb{N}$ $|$ $j < x$ $\}$.

\item $\mathbb{Q}$ is defined to be the set of rational numbers.
\end{itemize}
\end{tcolorbox}





\subsection{Assumed Knowledge}
\label{subsection:assumed_knowledge}


\subsubsection{Remainders \& Divisibility}
\label{subsubsection:remainders_and_divisibility}

\label{theorem:remainder_theorem}
\hypertarget{theorem:remainder_theorem}{}
\begin{tcolorbox}
\textbf{Remainder Theorem}

\noindent For any natural number $x$, for any positive integer $y$, there exists a unique integers $z \in \mathbb{N}$ and $r \in [y]$ such that $x = zy + r$. We call $r$ the \textit{remainder} of $x$ when divided by $z$.
\end{tcolorbox}
\noindent \\








\label{definition:remainder_function}
\hypertarget{definition:remainder_function}{}
\begin{tcolorbox}
\textbf{Definition}

\noindent For any natural number $x$, for any positive integer $y$,  the remainder function $\rho : (\mathbb{N} \times \mathbb{N}^+) \rightarrow \mathbb{N}$, $\rho(x, y)$ is defined to be the remainder of $x$ when divided by $y$.
\end{tcolorbox}
\noindent \\





\label{remainder_properties}
\hypertarget{remainder_properties}{}
\begin{tcolorbox}
\textbf{Remainder Function Properties}

\noindent The remainder function has the following properties:

\begin{itemize}

\item For any $i \in \mathbb{N}^+$, for any $j \in [i]$, $\rho(j, i) = j$.

\item For any $i \in \mathbb{N}^+$, for any $j \in \mathbb{Z}$, $\rho(ij, j) = 0$.

\item For any $j, k \in \mathbb{N}, i \in \mathbb{N}^+$, $\rho(kj, ki) = k \rho(j, i)$.

\item For any $j, k \in \mathbb{N}, i \in \mathbb{N}^+$, $\rho(j + k, i) = \rho(\rho(j, i) + \rho(k, i), i)$. 

\item For any $j, k \in \mathbb{N}, i \in \mathbb{N}^+$, $\rho(k, i) = k - ji$ whenever \\$ji \leq k < (j + 1)i$. 

\end{itemize}
\end{tcolorbox}
\noindent \\







\label{definition:divisibility}
\hypertarget{definition:divisibility}{}
\begin{tcolorbox}
\textbf{Definition}

\noindent For any natural number $x$, for any positive integer $y$, if $\rho(x,y) = 0$, then we say that $x$ is \textit{divisible} by $y$, that $x$ is a \textit{multiple} of $y$, and that $y$ is a \textit{divisor} of $x$.
\end{tcolorbox}
\noindent \\





\label{lemma:remainder_powers}
\hypertarget{lemma:remainder_powers}{}
\begin{tcolorbox}
\textbf{Lemma}

\noindent For any positive integers $x, y \geq 2$, such that $\rho(x,y) = 0$, it is the case that there exists some integers $z, p \in \mathbb{N}^+$ such that $x = zy^p$, $z < x$ and $\rho(z, y) > 0$.\\

\noindent \hyperlink{proof:remainder_powers}{Go to proof}

\end{tcolorbox}




\label{definition:common_multiple}
\hypertarget{definition:common_multiple}{}
\begin{tcolorbox}
\textbf{Definition}

\noindent For any set of integers $X \subset \mathbb{N}^+$, an integer $z$ is called a \textit{common multiple} of the elements of $X$ whenever $z$ is a multiple of every element of $X$.
\end{tcolorbox}
\noindent \\








\subsubsection{Miscellaneous}
\label{subsubsection:miscellaneous}




\label{definition:ceiling_function}
\hypertarget{definition:ceiling_function}{}
\begin{tcolorbox}
\textbf{Definition}

\noindent For any $q \in \mathbb{Q}$, the \textit{ceiling function} $\ceil{q} = z$, where $z$ is the integer such that $z - 1 < q \leq z$.
\end{tcolorbox}
\noindent \\








\label{lemma:ceiling_function}
\hypertarget{lemma:ceiling_function}{}
\begin{tcolorbox}
\textbf{Lemma}

\noindent For any $q = \frac{a}{b} \in \mathbb{Q}$, $a, b \in \mathbb{Z}$

\begin{itemize}
\item $\ceil{q} = q$ whenever $\rho(a, b) = 0$.

\item  $\ceil{q} = \frac{c}{b}$, where $c = a + b - \rho(a, b)$ whenever $\rho(a, b) > 0$.
\end{itemize}


\end{tcolorbox}
\noindent \\








\label{definition:gcd}
\hypertarget{definition:gcd}{}
\begin{tcolorbox}
\textbf{Definition}

\noindent For any $x, y \in \mathbb{N}$, the function $gcd(x, y)$ is defined to be the greatest common divisor of $x$ and $y$.
\end{tcolorbox}
\noindent \\








\label{definition:prime_numbers}
\hypertarget{definition:prime_numbers}{}
\begin{tcolorbox}
\textbf{Definition}

\noindent Any integer $p > 1$ is called \textit{prime} if its only divisors are one and itself.

\end{tcolorbox}
\noindent \\









\section{Awkward Number Series}
\label{section:awkward_number_series}


\subsection{Definition \& Basic Properties}

\label{definition:awkward_number_series}
\hypertarget{definition:awkward_number_series}{}
\begin{tcolorbox}
\textbf{Definition}

\noindent For any positive integers $a, n$, the \textit{awkward number series}, $S_{a, n}$ is defined as:

\begin{itemize}
\item An initial element $s_0 = a + n$
\item For any $i > 0$, $s_i$ is defined to be the least greatest integer such that $s_i > s_{i - 1}$ and $\rho(s_i, s_k) \geq a$ so all $k < i$.
\end{itemize}


\noindent We say that the awkward number series $S_{a,n}$ has $a$ \textit{activators}, and $n$ \textit{initial non-activators}.

\end{tcolorbox}
\noindent \\









\label{lemma:exists_element_less_than_x}
\hypertarget{lemma:exists_element_less_than_x}{}
\begin{tcolorbox}
\textbf{Lemma}

For any awkward number series $S_{a, n}$, for any $x \in \mathbb{N}$ such that $x > a + n$, $x$ is either an element of $S_{a, n}$ or there exists some $s_j \in S_{a, n}$ such that $x > s_j$ and $\rho(x, s_j) < a$. 
\end{tcolorbox}

\noindent \\
\textit{Proof}

\noindent Let $S_{a, n}$ be any awkward number series.\\

\noindent Let $x \in \mathbb{N}$ be any natural number such that $x > a + n$.\\

\noindent Assume that there does not exist an $s_j \in S_{a, n}$ such that $s_j < x$ and $\rho(x, s_j) < a$.\\

\noindent As such, for all $s_j \in S_{a, n}$ such that $s_j < x$, $\rho(x, s_j) \geq a$ must be the case.\\

\noindent \hyperlink{definition:awkward_number_series}{By definition} $x$ must be an element of $S_{a, n}$.\\

\noindent Now let us assume there exists some element $s_j \in S_{a, n}$ such that $s_j < x$ and $\rho(s_j, x) < a$.\\

\noindent As such, it is not the case that for all $s_j \in S_{a, n}$ such that $s_j < x$, $\rho(x, s_j) \geq a$.\\

\noindent \hyperlink{definition:awkward_number_series}{By definition} $x$ cannot be an element of $S_{a, n}$.


\begin{center}
\noindent\rule{8cm}{0.4pt}
\end{center}
\noindent \\









\label{lemma:non_divisibility_of_elements}
\hypertarget{lemma:non_divisibility_of_elements}{}
\begin{tcolorbox}
\textbf{Lemma}

For any awkward number series $S_{a, n}$, for any $s_i, s_j \in S_{a, n}$, $\rho(s_i, s_j) < a$ if and only if $s_i = s_j$.
\end{tcolorbox}

\noindent \\
\textit{Proof}

\noindent Let $S_{a, n}$ be any awkward number series.\\

\noindent Let $s_i \in S_{a, n}$ be any element in the series.\\

\noindent $s_i = s_i + 0$. As such, $\rho(s_i, s_i) = 0$ \hyperlink{theorem:remainder_theorem}{by definition} of the remainder.\\

\noindent \hyperlink{definition:awkward_number_series}{By definition} of an awkward number series, $a \geq 1 > 0$.\\

\noindent Let $s_j \in S_{a, n}$ be any element of the series such that $s_j < s_i$.\\

\noindent \hyperlink{definition:awkward_number_series}{By definition} of an awkward number series, $\rho(s_i, s_j) \geq a$.\\

\noindent As such, $\rho(s_i, s_j) < a$ cannot be the case.\\

\noindent Let $s_k \in S_{a, n}$ be any element of the series such that $s_k > s_i$.\\

\noindent $s_i = 0s_k + s_i$. As such, $\rho(s_i, s_k) = s_i$ \hyperlink{theorem:remainder_theorem}{by definition} of the remainder.\\

\noindent \hyperlink{definition:awkward_number_series}{By definition} of an awkward number series, $s_i \geq s_0 = a + n$.\\

\noindent As such, $\rho(s_i, s_k) = s_i \geq a + n \geq a$.

\begin{center}
\noindent\rule{8cm}{0.4pt}
\end{center}
\noindent \\






\label{corollary:non_divisibility_of_elements}
\hypertarget{corollary:non_divisibility_of_elements}{}
\begin{tcolorbox}
\textbf{Corollary}

For any awkward number series $S_{a, n}$, for any $s_i \in S_{a, n}$, it is the case that $s_{i + 1} \geq s_i + a$.
\end{tcolorbox}

\noindent \\
\textit{Proof}

\noindent Let $S_{a, n}$ be any awkward number series.\\

\noindent Let $s_i \in S_{a, n}$ be any element within the series.\\

\noindent Assume $s_{i + 1} < s_i + a$.\\

\noindent Subtracting $s_i$ from both sides yields, $s_{i + 1} - s_i < a$.\\

\noindent \hyperlink{definition:awkward_number_series}{By definition}, $s_i < s_{i + 1}$. As such, $s_{i + 1} - s_i > 0$.\\

\noindent Let $r = s_{i + 1} - s_i$. Then $0 < r < a < s_i$.\\

\noindent Furthermore, $s_{i + 1} = s_i + (s_{i - 1} - s_i) = s_i + r$.\\

\noindent \hyperlink{theorem:remainder_theorem}{By definition}, $r$ must be the remainder of $s_{i + 1}$ when divided by $s_i$.\\

\noindent As such, $\rho(s_{i + 1}, s_i) = r < a$. However, this contradicts \hyperlink{lemma:non_divisibility_of_elements}{the previous lemma}.

\begin{center}
\noindent\rule{8cm}{0.4pt}
\end{center}
\noindent \\






\label{theorem:infinite_asn}
\hypertarget{theorem:infinite_asn}{}
\begin{tcolorbox}
\textbf{Awkward Infinity Theorem}

Every awkward number series contains an infinite number of elements.
\end{tcolorbox}

\noindent \\
\textit{Proof}

\noindent Let $S_{a,n}$ be any awkward number series.\\

\noindent Assume that $S_{a,n}$ contains a finite number of elements.\\

\noindent Let $s_{max}$ be the greatest element within $S_{a, n}$.\\

\noindent Let $m$ be any common multiple of the elements of $S_{a, n}$ such that $m > s_{max}$.\\

\noindent Consider the value $m + a$.\\

\noindent By assumption $S_{a, n}$ is finite, as such, for any integer $x > s_{max}$, there exists some $s_j \in S_{a,n}$ such that $\rho(m + a, s_j) < a$ \hyperlink{lemma:exists_element_less_than_x}{by previous lemma}.\\

\noindent Let $\rho(m + a, s_j) = b < a$.\\

\noindent By the \hyperlink{theorem:remainder_theorem}{remainder theorem}, there exists some integer $x$ such that\\ $m + a = xs_j + b$.\\

\noindent Since $m$ is a common multiple of all the elements of $S_{a,n}$, then $\frac{m}{s_j} \in \mathbb{N}$.\\

\noindent Let $y = \frac{m}{s_j}$. Then $m = ys_j$.\\

\noindent Consider the equation $a = (m + a) - m$.\\

\noindent Substituting $xs_j + b$ for $m + a$ yields $a = xs_j + b - m$.\\

\noindent Substituting $ys_j$ for $m$ yields $a = xs_j + b - ys_j$.\\

\noindent Applying the distributive property yields $a = (x - y)s_j + b$.\\

\noindent Since $b < a < s_j$, then $b$ must be the remainder of $a$ when divided by $s_j$ \hyperlink{theorem:remainder_theorem}{by definition}, as such $\rho(a, s_j) = b$.\\

\noindent Furthermore, $a < s_j$, as such $\rho(a, s_j) = a = b$ \hyperlink{remainder_properties}{by properties of $\rho$}.\\

\noindent However, $b < a$ by assumption. As such, we have reached a contradiction.\\

\noindent Therefore, it must be the case that either $m + a$ is an element of $S_{a,n}$, or there exists some other element in $S_{a,n}$ less than $m + a$ that was not accounted for. In either case, $S_{a,n}$ cannot be finite.


\begin{center}
\noindent\rule{8cm}{0.4pt}
\end{center}
\noindent \\






\subsection{Prime Numbers}

\label{lemma:asn_subset_prime}
\hypertarget{lemma:asn_subset_prime}{}
\begin{tcolorbox}
\textbf{Lemma}

Every element of $S_{1, 1}$ is prime. 
\end{tcolorbox}

\noindent \\
\textit{Proof}

\noindent Assume there exists $s_i \in S_{1, 1}$ such that $s_i$ is not prime.\\

\noindent Then there exists integers $u, v$ such that $1 < u \leq v < s_i$, $s_i = uv$.\\

\noindent Assume there exists $s_j < s_i$ such that $s_j$ divides either $u$ or $v$.\\

\noindent Then $u = ts_j$ or $v = ts_j$ for some integer $t$.\\

\noindent As such, $s_i = ts_jv$ or $s_i = uts_j$.\\

\noindent In either case, $\rho(s_i, s_j) = 0$.\\

\noindent However, \hyperlink{definition:awkward_number_series}{by definition}, $\rho(s_i, s_j) > 0$.\\

\noindent As such, it must be the case that $\rho(u, s_k) \geq 1$ for all $s_k < s_i$.\\

\noindent Let $s_j \in S_{1, 1}$ be the element such that $s_j < u < s_{j + 1}$.\\

\noindent However, $s_{j + 1}$ is the least greatest integer greater than $s_j$ with the property that $\rho(s_{j + 1}, s_k) \geq 1$ for all $s_k \leq s_j$.\\

\noindent As such, $u$ cannot exist. Therefore, the only divisors of $s_i$ are $1$ and itself.\\

\noindent Thus, $s_i$ is prime \hyperlink{definition:prime_numbers}{by definition}.

\begin{center}
\noindent\rule{8cm}{0.4pt}
\end{center}
\noindent \\






\label{lemma:primes_in_asn}
\hypertarget{lemma:primes_in_asn}{}
\begin{tcolorbox}
\textbf{Lemma}

For any $n \in \mathbb{N}^+$, for any prime $p \geq 1 + n$, it is the case that $p \in S_{1, n}$.
\end{tcolorbox}

\noindent \\
\textit{Proof}

\noindent Let $n$ be any positive integer. Let $p$ be any prime such that $p \geq 1 + n$.\\

\noindent Note that $1 + n = s_0 \in S_{1, n}$ \hyperlink{definition:awkward_number_series}{by definition of an awkward number series}.\\

\noindent Let $s_i \in S_{1, n}$ such that $s_i \leq p < s_{i + 1}$.\\

\noindent Assume $p > s_i$.\\

\noindent \hyperlink{definition:prime_numbers}{By definition of prime}, the only factors of $p$ are $1$ and $p$.\\

\noindent As such, $\rho(p, s_j) \geq 1$ for all $s_j \in S_{1, n}$ such that $s_j < p$.\\

\noindent By assumption, $s_i < p$, as such, $\rho(p, s_j) \geq 1$ for all $s_j \leq s_i$.\\

\noindent \hyperlink{definition:awkward_number_series}{By definition of an awkward number series}, $s_{i + 1}$ is the least greatest integer greater than $s_i$ such that $\rho(p, s_j) \geq 1$ for all $s_j \leq s_i$.\\

\noindent As such, $p = s_{i + 1}$ must be the case.\\

\noindent However, $s_{i + 1}$ was chosen such that $p < s_{i + 1}$. As such, we have reached a contradiction. Therefor, it must be the case that $p = s_i$.

\begin{center}
\noindent\rule{8cm}{0.4pt}
\end{center}





\label{lemma:prime_asn}
\hypertarget{lemma:prime_asn}{}
\begin{tcolorbox}
\textbf{Lemma}

The awkward number series $S_{1, 1}$ is equal to the set of prime numbers.
\end{tcolorbox}

\noindent \\
\textit{Proof}

\noindent \hyperlink{lemma:asn_subset_prime}{By the previous lemma}, we know that the elements of $S_{1, 1}$ are a subset of the prime numbers. As such, we need to show that every prime is an element of $S_{1, 1}$.\\

\noindent \hyperlink{lemma:primes_in_asn}{By previous lemma}, $S_{1, 1}$ contains every prime greater than or equal to $1 + 1 = 2$.\\

\noindent \hyperlink{definition:prime_numbers}{By definition of primes}, prime numbers are integers strictly greater than $1$. As such, every prime is greater than or equal to $2$.\\

\noindent As such, $S_{1, 1}$ contains every prime number.

\begin{center}
\noindent\rule{8cm}{0.4pt}
\end{center}
\noindent \\





\label{corollary:infinite_primes}
\hypertarget{corollary:infinite_primes}{}
\begin{tcolorbox}
\textbf{Corollary}

There are an infinite number of prime numbers.
\end{tcolorbox}

\noindent \\
\textit{Proof}

\noindent The set of prime numbers is equal to the elements of the awkward number series $S_{1, 1}$ \hyperlink{lemma:prime_asn}{by previous lemma}.\\

\noindent Every awkward number series contains an infinite number of elements \hyperlink{theorem:infinite_asn}{by the awkward infinity theorem}.

\begin{center}
\noindent\rule{8cm}{0.4pt}
\end{center}
\noindent \\








\subsection{Dimension, Staples, and Basis}




\label{definition:dimension}
\hypertarget{definition:dimension}{}
\begin{tcolorbox}
\textbf{Definition}

For any awkward number series $S_{a,n}$, the value $\ceil{\frac{n}{a}} + 1$ is called the \textit{dimension} of the series, denoted $dim(S_{a, n})$.

\end{tcolorbox}
\noindent \\







\label{definition:basis}
\hypertarget{definition:basis}{}
\begin{tcolorbox}
\textbf{Definition}

For any awkward number series $S_{a,n}$, for $i \in [dim(S_{a,n})]$, $s_i$ is called a \textit{basis} of the awkward number series.

\end{tcolorbox}
\noindent \\






\label{lemma:min_dimension}
\hypertarget{lemma:min_dimension}{}
\begin{tcolorbox}
\textbf{Lemma}

For any awkward number series $S_{a, n}$, it is the case that $dim(S_{a, n}) \geq 2$.

\end{tcolorbox}

\noindent \\
\textit{Proof}

\noindent Assume there exists an awkward number series $S_{a, n}$ such that $dim(S_{a, n}) < 2$.\\

\noindent \hyperlink{definition:dimension}{By definition}, $dim(S_{a, n}) = \ceil{\frac{n}{a}} + 1$.\\

\noindent \hyperlink{definition:awkward_number_series}{By definition}, $a, n \in \mathbb{N}^+$. As such, $\ceil{\frac{n}{a}} > 0$.\\

\noindent Adding $1$ to both sides yields $\ceil{\frac{n}{a}} + 1 = dim(S_{a, n}) > 1$.\\

\noindent \hyperlink{definition:ceiling_function}{By definition}, $\ceil{\frac{n}{a}} \in \mathbb{Z}$. As such, $dim(S_{a, n}) \geq 2$. 



\begin{center}
\noindent\rule{8cm}{0.4pt}
\end{center}








\label{lemma:basis_lengths}
\hypertarget{lemma:basis_lengths}{}
\begin{tcolorbox}
\textbf{Lemma}

For any awkward number series $S_{a,n}$, for any basis $s_i$ of the series, it is the case that $s_i = a(i + 1) + n$.
\end{tcolorbox}

\noindent \\
\textit{Proof}

\noindent Let $S_{a,n}$ be any awkward number series. We shall complete this proof via induction on the index of the first $\ceil{\frac{n}{a}} + 1$ elements.\\


\noindent
\textit{Base Case}

\noindent \hyperlink{definition:awkward_number_series}{By definition}, the initial element is $s_0 = a + n = a(0 + 1) + n$.\\


\noindent
\textit{Inductive Hypothesis}

\noindent Assume for the some integer $k$ such that $0 \leq k < \ceil{\frac{n}{a}}$, that $s_j = a(j + 1) + n$ for all $j \leq k$.\\

\noindent
\textit{Inductive Step}

\noindent Let $s_j$ be any element such that $s_j \leq s_k$.\\

\noindent By the inductive hypothesis, $s_k = a(k + 1) + n$ and $s_j = a(j + 1) + n$.\\

\noindent Redistributing the $a$ term in $s_k$ yields $a(k + 1) + n = a(j + 1) + (k - j)a + n$.\\

\noindent As such, $s_k = s_j + (k - j)a$ by substitution.\\

\noindent Adding $a$ to both sides yields $s_k + a = s_j + (k - j + 1)a$.\\

\noindent By the inductive hypothesis, $k < \ceil{\frac{n}{a}}$.\\

\noindent As such, $k - j + 1 < \ceil{\frac{n}{a}} - j + 1$.\\

\noindent Since $j > 0$, then $\ceil{\frac{n}{a}} - j + 1 < \ceil{\frac{n}{a}} + 1 \leq \ceil{\frac{n}{a}}$.\\

\noindent As such, $(k - j + 1)a \leq a\ceil{\frac{n}{a}}$.\\

\noindent If $\rho(n, a) > 0$, then $a\ceil{\frac{n}{a}} = \frac{c}{a}$ where $c = n + a - \rho(n, a)$ \hyperlink{lemma:ceiling_function}{by previous lemma}.\\

\noindent Since $\rho(n, a) > 0$, then $c < n + a = s_0$ \hyperlink{definition:awkward_number_series}{by definition}.\\

\noindent If $\rho(n, a) = 0$, then $a\ceil{\frac{n}{a}} = n < s_0$ \hyperlink{lemma:ceiling_function}{by previous lemma}.\\

\noindent In either case, $a\ceil{\frac{n}{a}} < s_0$.\\

\noindent As such, $(k - j + 1)a < s_0$, therefore, $(k - j + 1)a \in [s_j]$.\\

\noindent As such, since $s_{k + 1} = s_j + (k - j + 1)a$, then $\rho(s_{k + 1}, s_j) = (k - j + 1)a$.\\

\noindent Since $j \leq k$, then $(k - j + 1)a \geq (k - k + 1)a = a$.\\

\noindent As such, $\rho(s_k + a, s_j) \geq a$ for any $j \leq k$.\\

\noindent \hyperlink{corollary:non_divisibility_of_elements}{By previous corollary}, $s_{k + 1} \geq s_k + a$.\\

\noindent As such, $s_k + a$ is the least greatest integer greater than $s_k$ with the property that $\rho(s_k + a, s_j) \geq a$ for all $j \leq k$. Therefor, $s_{k + 1} = s_k + a$ \hyperlink{definition:awkward_number_series}{by definition}.\\

\noindent Substituting for $s_k$ yields, $s_{k + 1} = a(k + 1) + n + a = a(k + 2) + n$. As such, we have completed the inductive step.

\begin{center}
\noindent\rule{8cm}{0.4pt}
\end{center}




\label{definition:staple}
\hypertarget{definition:staple}{}
\begin{tcolorbox}
\textbf{Definition}

For any awkward number series $S_{a,n}$, $s_i \in S_{a, n}$ is called a \textit{staple} whenever $s_i = s_{i - 1} + a$.

\end{tcolorbox}
\noindent \\






\label{lemma:initial_staples}
\hypertarget{lemma:initial_staples}{}
\begin{tcolorbox}
\textbf{Lemma}

For any awkward number series $S_{a,n}$, for any integer $0 < i < dim(S_{a,n})$, the element $s_i \in S_{a,n}$ is a staple.

\end{tcolorbox}

\noindent \\
\textit{Proof}

\noindent Let $S_{a, n}$ be any awkward number series.\\

\noindent Let $i$ be any integer such that $0 < i < dim(S_{a, n})$.\\

\noindent \hyperlink{lemma:basis_lengths}{By previous lemma}, $s_i = a(i + 1) + n$ and $s_{i - 1} = ai + n$.\\

\noindent Consider the difference $s_i - s_{i - 1}$.\\

\noindent Substituting $a(i + 1) + n$ for $s_i$ yields $s_i - s_{i - 1} = a(i + 1) + n - s_{i - 1}$.\\

\noindent Substituting $ai + n$ for $s_{i - 1}$ yields $a(i + 1) + n - s_{i - 1} = a(i + 1) + n - (ai + n)$.\\

\noindent Distributing the $-1$ yields $a(i + 1) + n - (ai + n) = a(i + 1) + n - ai - n$.\\

\noindent Adding the $n$ terms yields, $a(i + 1) + n - ai - n = a(i + 1) - ai$.\\

\noindent Factoring the $a$ yields, $a(i + 1) - ai = a(i + 1 - i) = a(1) = a$.\\

\noindent As such, $s_i - s_{i - 1} = a$.\\

\noindent Adding $s_{i - 1}$ to both sides yields $s_i = s_{i - 1} + a$.\\

\noindent Thus, $s_i$ is a staple \hyperlink{definition:staple}{by definition}.



\begin{center}
\noindent\rule{8cm}{0.4pt}
\end{center}
\noindent \\





\label{lemma:dimension_a_greater_equal_n}
\hypertarget{lemma:dimension_a_greater_equal_n}{}
\begin{tcolorbox}
\textbf{Lemma}

\noindent For any awkward number series $S_{a,n}$ such that $a \geq n$, it is the case that $dim(S_{a, n}) = 2$.

\end{tcolorbox}

\noindent \\
\textit{Proof}

\noindent Let $S_{a, n}$ be an awkward number series such that $a \geq n$.\\

\noindent \hyperlink{definition:dimension}{By definition of dimension}, $dim(S_{a,n}) = \ceil{\frac{n}{a}} + 1$.\\

\noindent Since $a, n \in \mathbb{N}^+$ \hyperlink{definition:awkward_number_series}{by definition of an awkward number series}, then $\frac{n}{a} > 0$.\\

\noindent Since $n \leq a$, then $\frac{n}{a} \leq \frac{a}{a} = 1$.\\

\noindent As such, $0 < \frac{n}{a} \leq 1$, therefore, $\ceil{\frac{n}{a}} = 1$ \hyperlink{definition:ceiling_function}{by definition of the ceiling function}.\\

\noindent Adding $1$ to both sides yields $\ceil{\frac{n}{a}} + 1 = 2$.\\

\noindent Substituting in $dim(S_{a, n})$ yields $dim(S_{a,n}) = 2$.\\

\begin{center}
\noindent\rule{8cm}{0.4pt}
\end{center}
\noindent \\









\label{lemma:only_staple}
\hypertarget{lemma:only_staple}{}
\begin{tcolorbox}
\textbf{Lemma}

\noindent For any awkward number series $S_{a,n}$ such that $a \geq n$, the series only contains a single staple which is $s_1$.

\end{tcolorbox}

\noindent \\
\textit{Proof}

\noindent Let $S_{a, n}$ be an awkward number series such that $a \geq n$.\\

\noindent \hyperlink{lemma:dimension_a_greater_equal_n}{By previous lemma}, $dim(S_{a, n}) = 2$ since $a \geq n$.\\

\noindent Since $1 \in [2] = [dim(S_{a, n})]$, then $s_1$ is a staple \hyperlink{lemma:initial_staples}{by previous lemma}.\\

\noindent Now we must show that there can be no element $s_1 < s_i \in S_{a, n}$ that is also a staple.\\

\noindent Assume there exists some staple $s_i > s_1$.\\

\noindent \hyperlink{definition:staple}{By definition of a staple}, $s_i = s_{i - 1} + a$.\\

\noindent Since $s_i > s_1$, then $s_{i - 1} \geq s_1 > s_0$. As such, \hyperlink{definition:awkward_number_series}{by definition of an awkward number series}, $\rho(s_{i - 1}, s_0) \geq a$.\\

\noindent Furthermore, there exists some integer $t \in \mathbb{N}$ such that $s_{i - 1} = ts_0 + \rho(s_{i - 1}, s_0)$ \hyperlink{theorem:remainder_theorem}{by the remainder theorem}.\\

\noindent Adding $a$ to both sides yields $s_{i - 1} + a = ts_0 + \rho(s_{i - 1}, s_0) + a$.\\

\noindent Substituting in $s_i$ for $s_{i - 1} + a$ yields $s_i = ts_0 + \rho(s_{i - 1}, s_0) + a$.\\

\noindent Since $\rho(s_{i - 1}, s_0) \geq a$, then $\rho(s_{i - 1}, s_0) + a \geq a + a$.\\

\noindent Furthermore, $n \leq a$, as such, $\rho(s_{i - 1}, s_0) + a \geq a + a \geq a + n = s_0$.\\

\noindent Let $r = \rho(s_{i - 1}, s_0) + a - s_0$.\\

\noindent Since $\rho(s_{i - 1}, s_0) + a \geq s_0$, then $\rho(s_{i - 1}, s_0) + a - s_0 \geq 0$ by subtracting $s_0$ from both sides.\\

\noindent Substituting in $r$ yields $r \geq 0$.\\

\noindent Furthermore, both $\rho(s_{i - 1}, s_0) < s_0$ and $a < s_0$, as such,\\ $\rho(s_{i - 1}, s_0) + a < s_0 + s_0 = 2s_0$.\\

\noindent Subtracting $s_0$ from both sides yields, $\rho(s_{i - 1}, s_0) + a - s_0 < s_0$.\\

\noindent Substituting $r$ yields, $r < s_0$. As such, $0 \leq r < s_0$.\\

\noindent We have that $s_i = ts_0 + \rho(s_{i - 1}, s_0) + a$.\\

\noindent Since $s_0 - s_0 = 0$, then $s_i = ts_0 + \rho(s_{i - 1}, s_0) + a + (s_0 - s_0)$.\\

\noindent Substituting in $r$ yields, $s_i = ts_0 + r + s_0$.\\

\noindent Factoring $s_0$ yields, $s_i = (t + 1)s_0 + r$.\\

\noindent Since $r \in [s_0]$, then $\rho(s_i, s_0) = r$ \hyperlink{theorem:remainder_theorem}{by definition of the remainder}.\\

\noindent Since $\rho(s_{i - 1}, s_0) < s_0$, then $r = \rho(s_{i - 1}, s_0) + a - s_0 < s_0 + a - s_0 < a$ by substitution.\\

\noindent As such, $\rho(s_i, s_0) = r < a$. However, $\rho(s_i, s_0) \geq a$ \hyperlink{definition:awkward_number_series}{by definition of an awkward number series}. As such, we have reached a contradiction.\\

\noindent Therefor, our assumption that $s_i$ is a staple must be false. As such, there can be no staple greater than $s_1$.


\begin{center}
\noindent\rule{8cm}{0.4pt}
\end{center}
\noindent \\







\label{corollary:min_length_one_staple}
\hypertarget{corollary:min_length_one_staple}{}
\begin{tcolorbox}
\textbf{Corollary}

\noindent For any awkward number series $S_{a,n}$ such that $a \geq n$, for any $s_i, s_j \in S_{a, n}$ such that $s_1 \leq s_i < s_j$, it is the case that $s_j \geq (j - i)(a + 1) + s_i$.

\end{tcolorbox}

\noindent \\
\textit{Proof}

\noindent Let $S_{a, n}$ be any awkward number series such that $a \geq n$.\\

\noindent We shall complete this proof by induction on the difference of indexes $i$ and $j$ for elements $s_i, s_j \in S_{a, n}$.

\noindent \\
\textit{Base Case}

\noindent Let $i \geq 1$ and $j = i + 1$.\\

\noindent As such, $j \geq 1 + 1 = 2$ by substitution.\\

\noindent Therefor, $s_j$ cannot be a staple \hyperlink{lemma:only_staple}{by previous lemma} since $a \geq n$ and $j \geq 2$.\\

\noindent As such, $s_j > s_i + a$. Since $s_j \in \mathbb{Z}$, then $s_j \geq s_i + a + 1$.\\

\noindent Furthermore, $j - i = (i + 1) - i = 1$ by substitution.\\

\noindent As such, $s_j \geq s_i + a + 1 = s_i + (1)(a + 1) = s_i + (j - i)(a + 1)$.


\noindent \\
\textit{Inductive Hypothesis}

\noindent Assume for some integer $k$ such that $1 \leq k$, that $s_j \geq s_i + (j - i)(a + 1)$ whenever $s_j > s_i$ and $j - i \leq k$.


\noindent \\
\textit{Inductive Step}

\noindent Let $s_i \in S_{a, n}$ such that $s_1 \leq s_i$.\\

\noindent Since $k = (k + i) - i$, then the element $s_{i + k} \geq s_i + k(a + 1)$ by the inductive hypothesis.\\

\noindent Since $i, k \geq 1$, then $i + k + 1 \geq 1 + 1 + 1 = 3$ by substitution.\\

\noindent As such, $s_{i + k + 1} \geq s_3 > s_1$. Therefore, $s_{i + k + 1}$ cannot be a staple \hyperlink{lemma:only_staple}{by previous lemma}.\\

\noindent As such, $s_{i + k + 1} > s_{i + k} + a$.\\

\noindent Since $s_{i + k + 1}, s_{i + k}, a \in \mathbb{Z}$, then $s_{i + k + 1} \geq s_{i + k} + a + 1$.\\

\noindent Substituting out $s_{i + k}$ yields, $s_{i + k + 1} \geq k(a + 1) + s_i + (a + 1)$.\\

\noindent Factoring the $(a + 1)$ yields $s_{i + k + 1} \geq (k + 1)(a + 1) + s_i$.\\

\noindent Furthermore, $(i + k + 1) - i = k + 1$. As such, $s_{i + k + 1} \geq ((i + k + 1) - i)(a + 1) + s_i$ by substitution.

\begin{center}
\noindent\rule{8cm}{0.4pt}
\end{center}
\noindent \\






\label{corollary:min_length_one_staple_pt2}
\hypertarget{corollary:min_length_one_staple_pt2}{}
\begin{tcolorbox}
\textbf{Corollary}

\noindent For any awkward number series $S_{a,n}$ such that $a \geq n$, for any $s_i \in S_{a, n}$ such that $s_1 \leq s_i$, it is the case that $s_i \geq (i - 1)(a + 1) + 2a + n$.

\end{tcolorbox}

\noindent \\
\textit{Proof}

\noindent Let $S_{a, n}$ be any awkward number series such that $a \geq n$.\\

\noindent Let $s_i$ be any element of $S_{a, n}$ such that $s_i \geq s_1$.\\

\noindent \hyperlink{corollary:min_length_one_staple_pt2}{By the previous corollary}, $s_i \geq (i - 1)(a + 1) + s_1$ since $s_i \geq s_1$.\\

\noindent \hyperlink{lemma:dimension_a_greater_equal_n}{By previous lemma}, $dim(S_{a, n}) = 2$ since $a \geq n$.\\

\noindent As such, $s_1$ is a basis \hyperlink{definition:basis}{by definition}.\\

\noindent As such, $s_1 = (1 + 1)a + n = 2a + n$ \hyperlink{lemma:basis_lengths}{by previous lemma} since $s_1$ is a basis.\\

\noindent Substituting out $s_1$ yields, $s_i \geq (i - 1)(a + 1) + 2a + n$.


\begin{center}
\noindent\rule{8cm}{0.4pt}
\end{center}
\noindent \\


\subsection{Linearity Theorem}



\label{lemma:relation_to_other}
\hypertarget{lemma:relation_to_other}{}
\begin{tcolorbox}
\textbf{Lemma}

For any awkward number series $S_{a,n}$, for any $i > 0$, there exists $s_j < s_i$ such that $\rho(s_i, s_j) = a$.

\end{tcolorbox}

\noindent \\
\textit{Outline}

\noindent This will be a proof by contradiction. We will assume that there exists some element $s_i \in S_{a,n}$, $s_0 < s_i$ such that $\rho(s_i, s_j) \neq a$ for all $s_j < s_i$. We will see this must mean that $s_{i - 1} = s_i - 1$. Finally we will find that this implies that $\rho(s_{i - 1}, s_i) \leq a$ which contradicts the definition of an awkward number series.

\noindent \\
\textit{Proof}

\noindent Let $S_{a, n}$ be any awkward number series.\\

\noindent Assume that there exists $s_i \in S_{a, n}$, $s_0 < s_i$ such that for all $s_j < s_i$, $\rho(s_i, s_j) \neq a$.\\

\noindent \hyperlink{definition:awkward_number_series}{By definition}, we know that $\rho(s_i, s_j) \geq a$.\\

\noindent As such, it must be the case that $\rho(s_i, s_j) > a$ since $\rho(s_i, s_j) \neq a$ by assumption.\\

\noindent Let $\rho(s_i, s_j) = r$.\\

\noindent $s_i = ts_j + r$ for some integer $t \in \mathbb{N}$ by \hyperlink{theorem:remainder_theorem}{the remainder theorem}.\\

\noindent Subtracting $1$ from both sides yields $s_i - 1 = ts_j + (r - 1)$.\\

\noindent Since $a < r$ and $a \in \mathbb{Z}$, then $a \leq r - 1$.\\

\noindent Furthermore, $r - 1 < r < s_j$, thus $r - 1 \in [s_j]$.\\

\noindent \hyperlink{theorem:remainder_theorem}{By definition}, $r - 1$ must be the remainder of $s_i$ when divided by $s_j$.\\

\noindent As such, for all $s_j < s_i$, $\rho(s_i, s_j) = r - 1 \geq a$.\\

\noindent This implies that $s_i - 1 = s_{i - 1}$ \hyperlink{definition:awkward_number_series}{by definition}.\\

\noindent By assumption, $s_i$ has a remainder strictly greater than $a$ when divided by any element less than it. As such, $\rho(s_i, s_{i - 1}) > a$ must be the case.\\

\noindent Substituting $s_i$ with $s_{i - 1} + 1$ yields $\rho(s_i, s_{i - 1}) = \rho(s_{i - 1} + 1, s_{i - 1})$.\\

\noindent \hyperlink{remainder_properties}{By remainder property}, $\rho(s_{i - 1} + 1, s_{i - 1}) = \rho(\rho(s_{i - 1}, s_{i - 1}) + \rho(1, s_{i - 1}), s_{i - 1})$.\\

\noindent \hyperlink{remainder_properties}{By remainder property}, $\rho(s_{i - 1}, s_{i - 1}) = 0$ since $s_{i - 1}$ is a multiple of itself.\\

\noindent \hyperlink{remainder_properties}{By remainder property}, $\rho(1, s_{i - 1})$ since $1 < s_{i - 1}$.\\

\noindent As such, $\rho(s_i, s_{i - 1}) = \rho(s_{i - 1} + 1, s_{i - 1}) = \rho(0 + 1, s_{i - 1}) = \rho(1, s_{i - 1}) = 1$.\\

\noindent \hyperlink{definition:awkward_number_series}{By definition}, $a \geq 1$. By assumption, $1 = \rho(s_i, s_{i - 1}) > a \geq 1$ which is a contradiction.


\begin{center}
\noindent\rule{8cm}{0.4pt}
\end{center}









\label{theorem:awkward_linearity}
\hypertarget{theorem:awkward_linearity}{}
\begin{tcolorbox}
\textbf{Awkward Linearity Theorem}

For any awkward number series $S_{a,n}$, for any $s_i \in S_{a,n}$, there exists integers $x, y \in \mathbb{N}^+$ such that $s_i = xa + yn$.

\end{tcolorbox}

\noindent \\
\textit{Proof}

\noindent This shall be a proof by induction. Let $S_{a,n}$ be any awkward number series.


\noindent \\
\textit{Base Case}

\noindent By definition, $s_0 = a + n = 1a + 1n$.


\noindent \\
\textit{Inductive Hypothesis}

\noindent Assume for some $0 \leq k$, that $s_i = xa + yn$ for some $x, y \in \mathbb{N}^+$ whenever $i \leq k$.


\noindent \\
\textit{Inductive Step}

\noindent By previous lemma, there exists some $s_i < s_{k+1}$ and some $t \in \mathbb{N}^+$ such that $s_{k+1} = ts_i + a$.\\

\noindent By the inductive hypothesis, $s_i = xa + yn$ for some integers $x, y \in \mathbb{N}^+$.\\

\noindent Substituting for $s_i$ yields, $s_{k+1} = t(xa + yn) + a = txa + a + yn = (tx + 1)a + yn$.\\



\begin{center}
\noindent\rule{8cm}{0.4pt}
\end{center}
\noindent \\







\label{corollary:relation_to_initial}
\hypertarget{corollary:relation_to_initial}{}
\begin{tcolorbox}
\textbf{Corollary}

For any awkward number series $S_{a,n}$, for any $s_0 < s_i \in S_{a,n}$, there exists integers $t, r \in \mathbb{N}^+$ such that $s_i = ts_0 + ra$.

\end{tcolorbox}


\noindent \\
\textit{Proof}

\noindent Let $S_{a, n}$ be any awkward number series. We shall complete this proof by induction.


\noindent \\
\textit{Base Case}

\noindent By previous lemma $s_1 = 2a + n = (a + n) + a = s_0 + a$.


\noindent \\
\textit{Inductive Hypothesis}

\noindent Assume for some $1 \leq k$, that $s_i = ts_0 + ra$ for some integers $t, r \in \mathbb{N}^+$ whenever $i \leq k$.


\noindent  \\
\textit{Inductive Step}

\noindent By previous lemma, there exists some $s_i < s_{k+1}$ and some $t \in \mathbb{N}^+$ such that $s_{k+1} = ts_i + a$.\\


\noindent If $s_i = s_0$, then we would have $s_{k+1} = ts_0 + a$. As such, we would have nothing left to show.\\

\noindent Let us assume $s_i > s_0$.\\

\noindent By inductive hypothesis, $s_i = us_0 + va$ for some integers $u, v \in \mathbb{N}^+$.\\

\noindent Substituting for $s_i$ yields, $s_{k + 1} = t(us_0 + va) + a = tus_0 + a(tv + 1)$


\begin{center}
\noindent\rule{8cm}{0.4pt}
\end{center}
\noindent \\








\label{corollary:relation_to_initial_p2}
\hypertarget{corollary:relation_to_initial_p2}{}
\begin{tcolorbox}
\textbf{Corollary}

For any awkward number series $S_{a,n}$, for any $s_0 < s_i \in S_{a,n}$, there exists integers $t, r \in \mathbb{N}^+$ such that $s_i = (t + r)a + tn$.

\end{tcolorbox}


\noindent \\
\textit{Proof}

\noindent Let $S_{a,n}$ be any awkward number series. Let $s_0 < s_i \in S_{a, n}$.\\

\noindent By previous corollary, $s_i = ts_0 + ra$ for some integers $t, r \in \mathbb{N}^+$.\\

\noindent Substituting for $a + n$ for $s_0$ yields, $s_i = t(a + n) + ra$.\\

\noindent Distributing $t$ over $a + n$ yields, $s_i = ta + ra + tn = (t + r)a + tn$.

\begin{center}
\noindent\rule{8cm}{0.4pt}
\end{center}
\noindent \\





\subsection{Uniqueness, Simplicity, and Similarity}



\label{theorem:awkward_uniqueness}
\hypertarget{theorem:awkward_uniqueness}{}
\begin{tcolorbox}
\textbf{Awkward Uniqueness Theorem}

\noindent For any two awkward number series $S_{a,b}$ and $S_{c,d}$, $S_{a,b} = S_{c,d}$ if and only if $a = c$ and $b = d$.\\

\noindent In other words, no two awkward series contain the same set of elements.
\end{tcolorbox}

\noindent \\
\textit{Proof}

\noindent Let $S_{a,n}$ be any awkward number series. Assume $S_{c, d} = S_{a, n}$ for some awkward number series $S_{c,d}$.\\


\noindent Let us refer to the elements of $S_{a,n}$ as $s_0, s_1, ...$, and the elements of $S_{c,d}$ by $s^*_0, s^*_1, ...$.\\

\noindent By definition, $s_0 = a + n$, and $s^*_0 = c + d$.\\

\noindent By assumption, $s_0 = s^*_0$. As such, $a + n = c + d$.\\

\noindent Solving for $c$ yields, $c = a + n - d$.\\

\noindent By previous lemma, $s_1 = 2a + n$, and $s^*_1 = 2c + d$.\\

\noindent By assumption, $s_1 = s^*_1$. As such, $2a + n = 2c + d$.\\

\noindent Substituting $c = a + n - d$ yields, $2a + n = 2(a + n - d) + d$.\\

\noindent Distributing the $2$ yields, $2a + n = 2a + 2n - 2d + d = 2a + 2n - d$.\\

\noindent Subtracting the $d$ from both sides yields, $2a + n + d = 2a + 2n$.\\

\noindent Subtracting the $2a$ from both sides yields $n + d = 2n$.\\

\noindent Subtracting $n$ from both sides yields $d = n$.\\

\noindent Substituting $n$ for $d$ into $a + n = c + d$ yields $a + n = c + n$.\\

\noindent Subtracting $n$ from both sides yields $a = c$.

\begin{center}
\noindent\rule{8cm}{0.4pt}
\end{center}










\label{definition:simple_and_redundant_series}
\hypertarget{definition:simple_and_redundant_series}{}
\begin{tcolorbox}
\textbf{Definition}

An awkward number series, $S_{a,n}$ is called \textit{simple} if the $gcd(a,n) = 1$. Otherwise the awkward number series is said to be \textit{redundant}.

\end{tcolorbox}
\noindent \\









\label{definition:similar_series}
\hypertarget{definition:similar_series}{}
\begin{tcolorbox}
\textbf{Definition}

Any two awkward number series $S_{a,b}$ and $S_{c,d}$ are called \textit{similar} whenever $\frac{a}{gcd(a,b)} = \frac{c}{gcd(c,d)}$ and $\frac{b}{gcd(a,b)} = \frac{d}{gcd(c,d)}$. Otherwise the series are said to be \textit{dissimilar}.

\end{tcolorbox}
\noindent \\







\label{theorem:similar_theorem}
\hypertarget{theorem:similar_theorem}{}
\begin{tcolorbox}
\textbf{Awkward Similarity Theorem}

\noindent For any simple awkward number series $S_{a, n}$, for any positive integer $x$, the elements of the awkward number series $S_{xa, xn} = \{$ $xs_i$ $|$ $s_i \in S_{a,n}$ $\}$.
\end{tcolorbox}

\noindent \\
\textit{Outline}

\noindent This will be a proof by induction on the index of the elements.

\noindent \\
\textit{Proof}

\noindent Let $S_{a,n}$ be any simple awkward number series. Let $j$ be any positive integer.\\

\noindent We shall denote the elements of $S_{a,n}$ as $s_0, s_1, ...$. We will denote the elements of $S_{ja, jn}$ as $s^*_0, s^*_1, ...$.

\noindent \\
\textit{Base Case}

\noindent By definition, the first element of $S_{ja, jn}$ is $s^*_0 = ja + jn = j(a + n)$.\\

\noindent By definition, the first element of $S_{a,n}$ is $s_0 = a + n$.\\

\noindent As such, $s^*_0 = j(a + n) = js_0$.



\noindent \\
\textit{Inductive Hypothesis}

\noindent Assume for all $0 \leq i$ that $s^*_i = js_i$.


\noindent \\
\textit{Inductive Step}

\noindent We shall start by showing that $\rho(js_{i+1}, s^*_k) \geq ja$ for all $k \leq i$. Afterwards, we will then show that $js_{i+1}$ is the least greatest integer that is both greater than $s^*_i$ with this property. As such, $s^*{i+1} = js^{i+1}$ by definition.

\noindent By the inductive hypothesis, $s^*_k = js_k$ for all $k \leq i$.\\

\noindent As such, $\rho(js_{i+1}, s^*_k) = \rho(js_{i+1}, js_k)$.\\

\noindent By previous lemma (TODO), $\rho(js_{i+1}, js_k) = j \rho(s_{i+1}, s_k)$.\\

\noindent By definition, $\rho(s_{i+1}, s_k) \geq a$ for all $k \leq i$.\\

\noindent As such, $\rho(js_{i+1}, s^*_k) = j \rho(s_{i+1}, s_k) \geq ja$.\\

\noindent Thus, we have shown that $js^{i+1}$ is a viable element of $S_{ja, jn}$. We now must show that that $js^{i+1}$ is the least greatest integer greater than $s^*_i$ with the divisibility property.\\

\noindent Assume there exists some integer $s^*_i < x < js^{i+1}$ such that $\rho(x, s^*_k) \geq ja$ for all $k \leq i$.\\

\noindent We know that $x = tj + \rho(x, j)$, for some $x \in \mathbb{N}$.

\noindent Let $r = \rho(x, j)$. Then $x = tj + r$.\\

\noindent Let $k \in [i + 1]$. Then $\rho(x, s^*_k) = \rho(tj + r, s^*_k)$.\\

\noindent By the inductive hypothesis, $s^*_k = js_k$.\\

\noindent As such, $\rho(tj + r, s^*_k) = \rho(tj + r, js_k)$\\

\noindent By remainder property (TODO), $\rho(tj + r, js_k) = \rho(\rho(tj, js_k) + \rho(r, js_k), js_k)$.\\

\noindent By remainder property (TODO), $\rho(tj, js_k) = j \rho(t, s_k)$.\\

\noindent Since $r < j < js_k$, then $\rho(r, js_k) = r$.\\

\noindent As such, $\rho(\rho(tj, js_k) + \rho(r, js_k), js_k) = \rho(j \rho(t, s_k) + r, js_k)$.\\

\noindent By definition, $0 \leq \rho(t, s_k) < s_k$. Furthermore, $\rho(t, s_k) \in \mathbb{N}$. As such, $\rho(t, s_k) \leq s_k - 1$.\\

\noindent As such, $j \rho(t, s_k) \leq j(s_k - 1)$.\\

\noindent Thus, $j \rho(t, s_k) + r \leq j(s_k - 1) + r$.\\

\noindent We also know that $r < j$.\\

\noindent As such, $j(s_k - 1) + r < j(s_k - 1) + j = j(s_k - 1 + 1) = js_k$.\\

\noindent As such, $j \rho(t, s_k) + r < js_k$, thus $\rho(x, s^*_k) = \rho(j \rho(t, s_k) + r, js_k) = j \rho(t, s_k) + r$.\\

\noindent By assumption, $\rho(x, s^*_k) \geq ja$.\\

\noindent As such, $j \rho(t, s_k) + j) > j \rho(t, s_k) + r \geq ja$.\\

\noindent As such, $j(\rho(t, s_k) + 1) > ja$.\\

\noindent As such, $\rho(t, s_k) + 1 > a$.\\

\noindent Thus, $\rho(t, s_k) \geq a$.\\

\noindent Now if we can show that $s_i < t < s_{i+1}$, then $t$ would have to be element $s_{i + 1} \in S_{a, n}$ which would be a contradiction.\\

\noindent By assumption, $s^*_i < x = jt + r$.\\

\noindent $s^*_i = js_i$ by the inductive hypothesis.\\

\noindent As such, $js_i < jt + r < jt + j = j(t + 1)$\\

\noindent Thus $s_i < t + 1$. Since $s_i \in \mathbb{N}$, then $s_i \leq t$.\\

\noindent However, we've shown that $\rho(t, s_i) \geq a > 0$. As such, $t \neq s_i$. Thus, $s_i < t$ must be the case.\\

\noindent Now we just need to show that $t < js_{i + 1}$.\\

\noindent We know that $x = tj + r < js_{i+1}$.\\

\noindent As such, $tj \leq tj + r < js_{i+1}$. Thus, $t < s_{i + 1}$.\\

\noindent But this would mean that $t$ must be the $(i + 2)^{th}$ element of $S_{a, n}$, which is a contradiction.


\begin{center}
\noindent\rule{8cm}{0.4pt}
\end{center}
\noindent \\













\subsection{Awkward Factorization}
\label{subsection:awkward_factorization}

\label{definition:awkward_factorization}
\hypertarget{definition:awkward_factorization}{}
\begin{tcolorbox}
\textbf{Definition}

For any $n \in \mathbb{N}^+$, for any awkward number series $S_{1,n}$, for any positive integer $x \geq 1 + n$. Whenever there exists some subset, $A$, of the elements of $S_{1,n}$ such that $x = cp$ for some integers $c$ and $p$ such that $1 \leq c < 1 + n$ and $p = \displaystyle \prod_{s_a \in A} s_a^{p_a}$ where $p_a \in \mathbb{N}^+$, then we call $cp$ an awkward factorization of $x$ in $S_{1, n}$.\\

\noindent For simplicity, whenever $c = 1$ we may exclude it from the awkward factorization of $x$. Similarly, we may exclude any of the powers $p_a$ from the awkward factorization if they are equal to $1$.

\end{tcolorbox}
\noindent \\






\label{lemma:awkward_factorization_of_element}
\hypertarget{lemma:awkward_factorization_of_element}{}
\begin{tcolorbox}
\textbf{Lemma}

For any $n \in \mathbb{N}^+$, for any awkward number series $S_{1,n}$, for any element $s_i \in S_{a, n}$, it is the case that $s_i$ is an awkward factorization of itself in $S_{1, n}$.
\end{tcolorbox}

\noindent \\
\textit{Proof}

\noindent Let $n \in \mathbb{N}^+$, let $s_i \in S_{1, n}$, and let $A = \{s_i\}$.\\

\noindent Then $s_i = (1)(s_i)^1$ is an awkward factorization of $s_i \in S_{1, n}$ \hyperlink{definition:awkward_factorization}{by definition}.


\begin{center}
\noindent\rule{8cm}{0.4pt}
\end{center}
\noindent \\




\label{lemma:non_common_factorization}
\hypertarget{lemma:non_common_factorization}{}
\begin{tcolorbox}
\textbf{Lemma}

For any $n \in \mathbb{N}^+$, for any awkward number series $S_{1,n}$, for any positive integer $x \geq 1 + n$, it is the case that there exists some integers $c, p \in \mathbb{N}^+$ and some element $s_t \in S_{1, n}$ such that $x = c(s_t)^p$, $c < x$, and $\rho(c, s_t) > 0$.

\end{tcolorbox}

\noindent \\
\textit{Proof}

\noindent Let $n \in \mathbb{N}^+$, let $x$ be any integer such that $x \geq 1 + n$.\\

\noindent If $x \in S_{1, n}$, then $x = 1(x) = 1(x)^1$ and we would be done since $\rho(1, x) = 1 > 0$ and $1 < 1 + n \leq x$.\\

\noindent As such, assume that $x$ is not an element of $S_{1, n}$.\\

\noindent \noindent \hyperlink{lemma:exists_element_less_than_x}{By previous lemma}, there exists some $s_t \in S_{1, n}$ such that $\rho(x, s_t) < 1$.\\

\noindent Since $\rho(x, s_t) \in [s_t]$ by definition, and $\rho(x, s_j) < 1$, then it must be the case that $\rho(x, s_j) = 0$.\\

\noindent Since $n \in \mathbb{N}^+$, then $n \geq 1$. As such, $x \geq 1 + n \geq 1 + 1 = 2$ by substitution.\\

\noindent Since $s_t \in S_{1, n}$, then $s_t \geq s_0 = 1 + n$ \hyperlink{definition:awkward_number_series}{by definition of an awkward number series}.\\

\noindent Furthermore, $s_t \geq 1 + n \geq 1 + 1 = 2$ by substitution.\\

\noindent As such, there exists integers $c, p \in \mathbb{N}^+$ such that $x = c(s_t)^p$, $c < x$, and $\rho(c, s_t) > 0$ \hyperlink{lemma:remainder_powers}{by previous lemma}.

\begin{center}
\noindent\rule{8cm}{0.4pt}
\end{center}
\noindent \\







\label{lemma:max_difference_of_elements}
\hypertarget{lemma:max_difference_of_elements}{}
\begin{tcolorbox}
\textbf{Lemma}

For any awkward number series $S_{a, n}$, for any element $s_i \in S_{a, n}$, it is the case that $s_{i + 1} \leq a + l$, where $l$ is the least common multiple of all elements $s_k \in S_{a, n}$ such that $s_k \leq s_i$.

\end{tcolorbox}

\noindent \\
\textit{Proof}

\noindent TODO: proof follows directly from the infinite theorem.


\begin{center}
\noindent\rule{8cm}{0.4pt}
\end{center}
\noindent \\






\label{lemma:closed_form_of_factorization}
\hypertarget{lemma:closed_form_of_factorization}{}
\begin{tcolorbox}
\textbf{Lemma}

For any $j \in \mathbb{N}^+$, if for all $i \in [j]$, $x_i = x_{i + 1}(a_i)^{p_i}$ such that $x_i, x_{i + 1}, a_i, p_i \in \mathbb{N}^+$, then for any $k, l \in [j]$ such that $k < l$, it is the case that $\displaystyle x_k = x_l \prod_{i = k}^{l - 1} (a_i)^{p_i}$
\end{tcolorbox}

\noindent\\
\textit{Proof}

\noindent We shall complete this proof by induction on the difference of the indexes.\\

\noindent Assume for all $i \in [j]$, for some integer $j \in \mathbb{N}^+$, that $x_i = x_{i + 1}(a_i)^{p_i}$ with $x_i, x_{i + 1}, a_i, p_i \in \mathbb{N}^+$.\\

\noindent\\
\textit{Base Case}

\noindent Let $k \in [j - 1]$. By assumption, $x_k = x_{k + 1}(a_k)^{p_k}$.\\

\noindent Furthermore, $\displaystyle \prod_{i = k}^{k}(a_i)^{p_i} = (a_k)^{p_k}$. As such, $\displaystyle x_k = x_{k + 1} \prod_{i = k}^{k}(a_i)^{p_i}$ by substitution.\\


\noindent \\
\textit{Inductive Hypothesis}

\noindent Assume for some integer $m$ such that $1 \leq m < j - 1$, that $\displaystyle x_k = x_l \prod_{i = k}^{l - 1} (a_i)^{p_i}$ whenever $l - k \leq m$.

\noindent \\
\textit{Inductive Step}

\noindent Let $k, l \in [j]$ be any integers such that $k - l = m + 1$.\\

\noindent Since $k - l = m + 1$, then we can subtract $1$ from both sides to yield $k - (l - 1) = m$. Furthermore, $m \geq 1$, so $k - (l - 1) \geq 1$ is also true.\\

\noindent As such, we can apply the inductive hypothesis: $\displaystyle x_k = x_{l - 1} \prod_{i = k}^{l - 2}(a_i)^{p_i}$.\\

\noindent By assumption, $x_{l - 1} = x_l(a_{l - 1})^{p_{l - 1}}$.\\

\noindent As such, $\displaystyle x_k = x_l(a_{l - 1})^{p_{l - 1}} \prod_{i = k}^{l - 2}(a_i)^{p_i} = x_l \prod_{i = k}^{l - 1}(a_i)^{p_i}$ by substitution.

\begin{center}
\noindent\rule{8cm}{0.4pt}
\end{center}
\noindent \\





\label{lemma:awkward_expansion}
\hypertarget{lemma:awkward_expansion}{}
\begin{tcolorbox}
\textbf{Lemma}

For any $n \in \mathbb{N}^+$, for any awkward number series $S_{1,n}$ for any $j \in \mathbb{N}^+$, if for all $i \in [j]$, $x_i = x_{i + 1}(a_i)^{p_i}$ where:

\begin{itemize}
\item $x_i, x_{i + 1}, p_i \in \mathbb{N}^+$

\item $x_i > x_{i + 1}$

\item $a_i \in S_{1, n}$ and $a_i = a_k$ if and only if $i = k$

\item $\rho(x_{i + 1}, a_k) > 0$ whenever $k \leq i$

\item $x_j \geq 1 + n$
\end{itemize}

then $x_j = x_{j + 1}(a_j)^{p_j}$ such that:

\begin{itemize}
\item $x_{j + 1}, p_j \in \mathbb{N}^+$

\item $x_j > x_{j + 1}$

\item $a_j \in S_{1, n}$ and $a_j \neq a_k$ for all $k < j$

\item $\rho(x_{j + 1}, a_k) > 0$ whenever $k \leq j$
\end{itemize}

\end{tcolorbox}

\noindent\\
\textit{Proof}

\noindent Assume for some integer $j$ that the properties described above hold.\\

\noindent Then there exists some integers $x, p \in \mathbb{N}^+$ and some element $a \in S_{1, n}$ such that $x_j = x(a)^p$, $x < x_j$, and $\rho(x, a) > 0$ \hyperlink{lemma:non_common_factorization}{by previous lemma}.\\

\noindent If we can show that $\rho(x, a_k) > 0$ for all $k \in [j]$, and $a \neq a_k$ for all $k \in [j]$, then we will have shown that $x = x_{j + 1}$, $a = a_j$, and $p = p_j$ and we will have completed our proof.\\



\noindent Let us begin by showing $\rho(x, a_k) > 0$ for all $k \in [j]$. We shall accomplish this by contradiction.\\

\noindent Assume that there exists $k \in [j]$ such that $\rho(x, a_k) = 0$.\\

\noindent As such, there exists some integer $u$ such that $x = ua_k$ \hyperlink{theorem:remainder_theorem}{by the remainder theorem}.\\

\noindent We shall now show that this leads to $\rho(x_{k + 1}, a_k) = 0$. In order to do so, we need to express $x_{k + 1}$ as a multiple of $a_k$.\\

\noindent There are two cases to consider, when $k = j - 1$ and $k < j - 1$. Let us start with the case where $k = j - 1$.\\

\noindent We know $x_{k + 1} = x_j = x(a)^p = ua_k(a)^p$ by substitution. Therefor, $\rho(x_j, a_k) = \rho(x_j, a_{j - 1}) = 0$. However, this contradicts our original assumption that $\rho(x_j, a_{j - 1}) > 0$. Therefor, $k < j - 1$ must be the case.\\

\noindent Now let us consider the case where $k < j - 1$.

\noindent We can apply \hyperlink{lemma:closed_form_of_factorization}{the previous lemma} to get $\displaystyle x_{k + 1} = x(a)^p \prod_{i = k + 1}^{j - 1} (a_i)^{p_i}$ .

\noindent As such, $\displaystyle x_{k+1} = ua_k(a)^p \prod_{i = k + 1}^{j - 1} (a_i)^p_i$ by substitution.\\

\noindent Therefor,  $\rho(x_{k + 1}, a_k) = 0$ \hyperlink{theorem:remainder_function}{by the definition of a remainder}.\\ 

\noindent However, we have that $k \in [j]$, thus our assumption $\rho(x_{k + 1}, a_k) > 0$ holds. As such, we have reached a contradiction by assuming $\rho(x, a_k) = 0$. Therefore, $\rho(x, a_k) > 0$ must actually be the case for all $k \in [j]$.\\



\noindent Now we are only left with showing that $a \neq a_k$ for all $k \in [j]$ to complete our proof. We shall once again use contradiction.\\

\noindent Assume there exists some integer $k \in [j]$ such that $a = a_k$.\\

\noindent We have $x_j = x(a)^p$. As such, $x_j = x(a)^p = x(a_k)^p$ by substitution.\\

\noindent Furthermore, $\displaystyle x_{k + 1} = x(a)^p \prod_{i = k + 1}^{j - 1} (a_i)^{p_i} = x(a_k)^p \prod_{i = k + 1}^{j - 1} (a_i)^{p_i}$ whenever $k + 1 < j$; and $x_{k + 1} = x_j = x(a_k)^p$ by substitution when $k + 1 = j$. In either case, $\rho(x_{k + 1}, a_k) = 0$.\\

\noindent However, this contradicts our assumption that $\rho(x_{k + 1}, a_k) > 0$. As such, our assumption that $a = a_k$ must have been incorrect. Therefore, $a \neq a_k$ for all $k \in [j]$ must hold.\\

\noindent As such, we have now shown that $x_j = x(a)^p$ where $x, p \in \mathbb{N}^+$, $x < x_j$, $a \in S_{1, n}$, $a \neq a_k$ for all $k \in [j]$, and $\rho(x, a_k) > 0$ as well as $\rho(x, a) > 0$.

\begin{center}
\noindent\rule{8cm}{0.4pt}
\end{center}
\noindent \\









\label{thereom:awkward_factorization}
\hypertarget{thereom:awkward_factorization}{}
\begin{tcolorbox}
\textbf{Awkward Factorization Theorem}

For any $n \in \mathbb{N}^+$, for any awkward number series $S_{1,n}$, for any positive integer $x$, it is the case that $x$ has an awkward factorization in $S_{1, n}$.
\end{tcolorbox}

\noindent\\
\textit{Outline}

\noindent In this proof we are going to construct the awkward factorization of any positive integer $x_0$.\\

\noindent We will do so by showing we can find a subset of elements of $A \subset S_{1, n}$ such that $x_0 = x_1(a_0)^{p_0}$, $x_1 = x_2(a_1)^{p_1}$, ..., $x_k = x_{k + 1}(a_k)^{p_k}$ where:
\begin{itemize}
\item $x_i \in \mathbb{N}^+$ and $x_{i - 1} < x_i$ for all $x_i$
\item $x_{k+1} \in [1 + n]$
\item $\rho(x_i, a_j) > 0$ for any $j < i$
\item $a_i \in A$ and $p_i \in \mathbb{N}^+$ for all $a_i$ and $p_i$
\end{itemize}

\noindent As such, we will have that:
\begin{center}
$x_0 = x_1(a_0)^{p_0} = x_2(a_1)^{p_1}(a_0)^{p_0} = ... = \displaystyle x_{k + 1} \prod_{i = 0}^k a_i^{p_i}$
\end{center}
which is an awkward factorization of $x_0$.


\noindent \\
\textit{Proof}

\noindent Let $n$ and $x_0$ be any positive integers.\\

\noindent If $x_0 \in [1 + n]$, then $x_0$ is the awkward factorization of itself and we are done.\\

\noindent As such, let us assume $x_0 \geq [1 + n]$.\\

\noindent As such, there exists some integers $x_1, p_0 \in \mathbb{N}^+$ and some element $a_0 \in S_{1, n}$ such that $x_0 = x_1(a_0)^{p_0}$, $x_1 < x_0$, and $\rho(x_1, a_0) > 0$ \hyperlink{lemma:non_common_factorization}{by previous lemma}.\\

\noindent If $x_1 < 1 + n$, then $x_1(a_0)^{p_0}$ is an awkward factorization and we are done.\\

\noindent As such, let us assume $x_1 \geq 1 + n$.\\

\noindent Applying the previous lemma yields $x_1 = x_2(a_1)^{p_1}$.\\

\noindent Once again, if $x_2 < 1 + n$ then $x = x_2(a_1)^{p_1}(a_0)^{p_0}$ would be an awkward factorization and we would be down.\\

\noindent Notice that we can repeatedly apply this argument to yield $x_i = x_{i + 1}(a_i)^{p_i}$ as long as $x_i > 1 + n$.\\

\noindent Let $s_t \in S_{1, n}$ be the element such that $s_t \leq x_0 < s_{t + 1}$.\\

\noindent Let $A = \{$ $a_i$ $|$ $i \in [t + 1]$ $\}$.\\

\noindent Since the elements of a



\noindent We shall now construct a repeatable argument to show that whenever\\ $x_k \geq 1 + n$ that we can find some $x_{k + 1}$ such that $x_i = x_{i + 1}(a_i)^{p_i}$ for all $i \in [k + 1]$ with $a_i \in S_{1, n}$ and $p_i \in \mathbb{N}^+$.\\



\noindent Assume the above argument was repeated such that we found $t + 1$ elements of $S_{1, n}$, such that $x_i = x_{i+1}(a_i)^{p_i}$ for all $i \in [t + 1]$.\\

\noindent Then $\displaystyle x_0 = x_{t + 1} \prod_{i = 0}^t (a_i)^{p_i}$.\\

\noindent As such, $a_i \leq x_0$ for all elements $a_i$.\\

\noindent Furthermore, $a_i \neq a_j$ unless $i = j$.\\

\noindent As such, by the pigeon hole principle, $\{ a_i$ $|$ $i \in [t + 1] \}$ $=$ $\{ s_i$ $|$ $i \in [t + 1]$.\\

\noindent As such, $\displaystyle x_0 = x_{t + 1} \prod_{i = 0}^t (a_i)^{p_i} = x_{t + 1} \prod_{i = 0}^t (s_i)^{q_i}$.\\

\noindent However, $\displaystyle x_0 < s_{t + 1} < lcm(s_i) + 1 \leq lcm(s_i) \leq \prod_{i = 0}^t (s_i)$.\\

\noindent As such, we have reached a contradiction. Therefor, it must be the case that we found some $x_k \in [1 + n]$ with $k \leq t$.\\

\noindent Therefore, $\displaystyle x_0 = x_{k} \prod_{i = 0}^{k - 1} (a_i)^{p_i}$ is an awkward factorization of $x_0$ by the elements of $S_{1, n}$.


\begin{center}
\noindent\rule{8cm}{0.4pt}
\end{center}
\noindent \\
















\subsection{Twins}
\label{subsection:twins}





\label{definition:twins}
\hypertarget{definition:twins}{}
\begin{tcolorbox}
\textbf{Definition}

For any awkward number series $S_{a,n}$, $s_{i - 1}, s_i \in S_{a,n}$ are called \textit{twins} whenever $s_i = s_{i - 1} + (a + 1)$.

\end{tcolorbox}
\noindent \\










\label{conjecture:twin_conjecture}
\hypertarget{conjecture:twin_conjecture}{}
\begin{tcolorbox}
\textbf{Awkward Twin Conjecture}

For any awkward number series $S_{a,n}$, $S_{a, n}$ contains an infinite number of twins.

\end{tcolorbox}
\noindent \\








\subsection{Proofs for Assumed Knowledge}
\label{subsection:assumed_knowledge_proofs}





\begin{tcolorbox}
\textbf{Lemma}

\noindent For any positive integers $x, y \geq 2$, such that $\rho(x,y) = 0$, it is the case that there exists some integers $z, p \in \mathbb{N}^+$ such that $x = zy^p$, $z < x$ and $\rho(z, y) > 0$.\\

\end{tcolorbox}


\noindent \\
\textit{Proof}
\label{proof:remainder_powers}
\hypertarget{proof:remainder_powers}{}

\noindent Let $x, y$ be any integers such that $x, y \geq 2$ and $\rho(x, y) = 0$.\\

\noindent If $x = y^p = (1)y^p$ for some $p \in \mathbb{N}^+$, then we would be done since $\rho(1, y) = 1$ and $1 < x$.\\

\noindent As such, assume $x \neq y^p$ for any $p \in \mathbb{N}^+$.\\

\noindent Since $\rho(x, y) = 0$, then there exists some integer $t \in \mathbb{N}^+$ such that $x = ty = ty^1$ by the remainder theorem.\\

\noindent Assume that for all $p \in \mathbb{N}^+$, that $\rho(x, y^p) = 0$.\\

\noindent Let $q$ be any integer such that $x < y^q$.\\

\noindent Then $\rho(x, y^q) = x$ since $x < y^q$ which contradicts $\rho(z, y^p) = 0$ for all $p \in \mathbb{N}^+$.\\

\noindent As such, there must exist some integer $p \in \mathbb{N}^+$ such that $\rho(x, y^p) = 0$ and $\rho(x, y^{p + 1}) \neq 0$.\\

\noindent By the remainder theorem, there exists some integer $z \in \mathbb{N}^+$ such that $x = zy^p$.\\

\noindent Assume that $\rho(z, y) = 0$.\\

\noindent Then there exists some integer $w$ such that $z = wy$ by the remainder theorem.\\

\noindent As such, $x = zy^p = wyy^p$ by substitution.\\

\noindent Furthermore, $x = wyy^p = wy^{p + 1}$ by properties of powers.\\

\noindent As such, $\rho(x, y^{p + 1}) = 0$ by definition of the remainder. However, we chose $p$ such that $\rho(x, y^{p + 1}) \neq 0$. Therefor, we have reached a contradiction and our assumption that $\rho(z, y) = 0$ must be false.\\

\noindent As such, $\rho(z, y) \neq 0$ must be the case.\\

\noindent Furthermore, since $y \geq 2$ then $y^p \geq 2 > 1$.\\

\noindent As such, $x = zy^p > z(1) = z$.


\begin{center}
\noindent\rule{8cm}{0.4pt}
\end{center}
\noindent \\





\end{document}






