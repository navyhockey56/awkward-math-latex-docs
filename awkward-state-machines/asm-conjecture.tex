\documentclass[a4paper,12pt]{article}
\usepackage{amsfonts}
\usepackage{tcolorbox}

\usepackage{hyperref}
\hypersetup{
	colorlinks=true,
	linkcolor=blue
}

\begin{document}

\label{definition:twin_awkward_numbers}
\hypertarget{definition:twin_awkward_numbers}{}
\begin{tcolorbox}
\textbf{Definition}

For any awkward number series $S_{a,n}$, consecutive elements are called \textit{twins} if their difference is $a + 1$.
\end{tcolorbox}





\label{conjecture:infinite_twins}
\hypertarget{conjecture:infinite_twins}{}
\begin{tcolorbox}
\textbf{Awkward Twins Conjecture}

For any awkward number series $S_{a,n}$, there are an infinite number of twin elements.
\end{tcolorbox}

\noindent \\
\textit{Note}
The twin prime conjecture which states there are an infinite number of twin primes has never been proven so I doubt you'll prove it as a corollary.




\label{definition:staple_awkward_number}
\hypertarget{definition:staple_awkward_number}{}
\begin{tcolorbox}
\textbf{Definition}

An element, $s_i$, of any awkward number series is called a \textit{staple} if it is $a$ less than the next element: $s_{i+1} - s_i = a$.
\end{tcolorbox}








\label{lemma:initial_staples}
\hypertarget{lemma:initial_staples}{}
\begin{tcolorbox}
\textbf{Lemma}

For any awkward number series $S_{a,n}$, the first $ceil(\frac{n}{a}) + 1$ elements are given by $s_i = a(i + 1) + n$.
\end{tcolorbox}

\noindent \\
\textit{Proof}

\noindent The initial element is given by $s_i = a(0 + 1) + n$.\\

\noindent Until the initial element completes its first cycle, the machine will be forced to create a new element every $a$ steps.

\begin{center}
\noindent\rule{8cm}{0.4pt}
\end{center}



\label{lemma:beyond_initial_staple}
\hypertarget{lemma:beyond_initial_staple}{}
\begin{tcolorbox}
\textbf{Lemma}

The $ceil(\frac{n}{a}) + 2 = i$ element is equal to $s_i = ai + 2n$ if $n > a$ or $s_i = a(i + 1) + 2n$ otherwise. 
\end{tcolorbox}

\noindent \\
\textit{Note}

\noindent If this is true, then this probably has something to do with the awkward twins conjecture.




\label{conjecture:staple_conjecture}
\hypertarget{conjecture:staple_conjecture}{}
\begin{tcolorbox}
\textbf{Staple Conjecture}

For any awkward number series $S_{a,n}$, the first $ceil(\frac{n}{a}) + 1$ elements are the only staples.
\end{tcolorbox}

\noindent \\
\textit{Note}

\noindent If this is true, then this probably has something to do with the awkward twins conjecture.









\label{conjecture:linear_combo_conjecture}
\hypertarget{conjecture:linear_combo_conjecture}{}
\begin{tcolorbox}
\textbf{Linear Combination Conjecture}

For any awkward number series $S_{a,n}$, every element can be expressed in the form $s_i = xa + yn$ for some integers $x, y \geq 0$.
\end{tcolorbox}

\noindent \\
\textit{Note}

\noindent I haven't the slightest clue why this is true... but all the examples I've run have held


\end{document}






